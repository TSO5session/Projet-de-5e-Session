\documentclass[10pt,a4paper,final]{article}

\usepackage{geometry}                            
\geometry{a4paper,total={210mm,297mm},left=10mm,right=10mm,top=20mm,bottom=20mm} 

\usepackage{fancyhdr}          % Met les pieds et têtes de pages
\pagestyle{fancy}

\usepackage[francais]{babel}   % Met les accents français
\usepackage[utf8]{inputenc}
\usepackage[pdftex]{graphicx}

\usepackage{graphicx}
\usepackage{framed}
\usepackage{tabularx}          % Liste les tableaux en table des matières
\usepackage{listings}
\usepackage{lipsum}
\usepackage{multicol}          % Permet le multicolone
\usepackage{xcolor}

\renewcommand\familydefault{\sfdefault}

\usepackage{tgheros}                
\usepackage[defaultmono]{droidmono} 
\usepackage{amsmath,amssymb,amsthm,textcomp}
%\usepackage{ubuntu}
   % Packages et dépendances nécessaires
\include{TeX/SourceCode} % Permet de mettre du code source dans le document

\begin{document}
\renewcommand\headrulewidth{0pt}
\fancyfoot[C]{ }






\begin{center}
\includegraphics[scale=1]{Figures/School_Logo.jpg}~\\[1cm]  
\textsc{\LARGE \'{E}lectronique Programmable et Robotique}\\[1.5cm]
\Large 247-637-LI\\[0.5cm]
{ \huge \bfseries Révision des fonctions de base \\[0.4cm] }
\HRule \\[1.5cm]





\begin{multicols}{2}
\begin{flushleft}



\textbf{Étudiants:}\\

\bigskip

Vincent Chouinard\\ 
\bigskip
Philippe Dubois\\




\end{flushleft}
\vfill
\begin{flushright}

\textbf{Professeur:}\\
\medskip
Ali Tadli\\



\end{flushright}
\end{multicols}

\bigskip
\bigskip
\bigskip

\includegraphics[scale=0.08]{Figures/Picture_for_Title.jpg} 

\vfill
Retour sur l'utilisation de la carte uPSD
\bigskip

{\large \today}
\end{center}






\pagebreak
\tableofcontents 
%\pagebreak
\listoffigures  
\listoftables   
\pagebreak






\renewcommand\headrulewidth{1pt}
\renewcommand\footrulewidth{1pt}
\fancyhead[L]{No. du cours}
\fancyhead[R]{Titre du projet}
\fancyfoot[C]{\textbf{page \thepage}}
\fancyfoot[R]{\today}
\fancyfoot[L]{V.C.}


%\section{Objectifs}
%\section{Objectifs}
Lorem ipsum dolor sit amet, consectetur adipiscing elit, sed do eiusmod tempor incididunt ut labore et dolore magna aliqua. Ut enim ad minim veniam, quis nostrud exercitation ullamco laboris nisi ut aliquip ex ea commodo consequat. Duis aute irure dolor in reprehenderit in voluptate velit esse cillum dolore eu fugiat nulla pariatur. Excepteur sint occaecat cupidatat non proident, sunt in culpa qui officia deserunt mollit anim id est laborum.\\

\textbf{Ce template permet:}
\begin{itemize}
\item[$\Rightarrow$]D'afficher directement du code source en C++ (d'autres langages pris en charge)
\item[$\Rightarrow$]D'afficher des tableaux et des figures listés en table des matières
\item[$\Rightarrow$]D'avoir une table des matières et une date qui s'auto-actualisent
\item[$\Rightarrow$]D'avoir les éléments habituels en pied et en tête de page
\item[$\Rightarrow$]D'avoir des marges généreuses pour mettre des schémas et des images
\item[$\Rightarrow$]D'avoir une police en Droid Sans avec possibilité de changement via les commentaires
\item[$\Rightarrow$]D'avoir le support intégral de la ponctuation francophone, y compris dans l'affichage de code source
\item[$\Rightarrow$]De basculer en mode multi colonne hétérogène \textit{on the fly}
\item[$\Rightarrow$]D'intégrer de l'art ASCII
\end{itemize}

\section{Expérimentation}
\subsection{Partie A}
\subsection{Partie B}
\subsection{Partie C}
\subsection{Partie D}
\subsection{Partie E}
\subsection{Partie F}

\section{Autre chose}
\subsection{Point A}
\subsection{Point B}
\subsection{Point C}

\pagebreak
























\section{Fonctionnalités \LaTeX{} avancées}

\subsection{Code source à partir d'un fichier externe}
\begin{center}
\fbox{\lstinputlisting[language=C++]{Codesource/exemple.h}}
\end{center}


\pagebreak

\subsection{Code source dans le fichier .Tex}
\begin{lstlisting}[label={list:first},caption=Code source en directe]
// **************************Exemple.h
// Auteur:      Vincent Chouinard
// Date:        11 mars 2014
// Version :    1.0
//
// Compilateur: IAR Embedded Workbench 8.1
//
// *****************************************************************************
#include "_DeclarationGenerale.h"  // Raccourcis de programmation & variables

#ifndef EXEMPLEH
  #define EXEMPLEH

class CLExemple : public CLSomethingElse
{
public:
   CLClasse(void);                 // ***Constructeur par défaut
   CLClasse(UINT16 usiAdresse);    // ***Constructeur initialisateur
  ~CLClasse(void);                 // ***Destructeur
  
  float VariableBabel_éÉàÀèÈïçîÎêÊ // Accents babel fonctionnels!

protected:
   void vFonction1(float);
   char cFonction2(printf("Hello World"));
   void vFonction3(UC, unsigned short int);

private:
   for(int i = 0; i <= 10; i++)
     {
      char   cVariable = cFonction2() -1;
      float  fVar;
      fVar++;
     }
};
#endif
//LinuxLinuxLinuxLinuxLinuxLinuxLinuxLinuxLinuxLinuxLinuxLinuxLinuxLinuxLinuxTUX

\end{lstlisting}
\pagebreak

\subsection{Formules mathématiques}
\begin{table}[ht]
%\caption{Multi-column table}
\begin{center}
\begin{tabular}{llll}  
    \hline
    \multicolumn{4}{c}{}\\
    
    \bigskip
    
    $V=RI$&$R=\dfrac{V}{I}$&$I=\dfrac{V}{R}$&$P=VI$\\
    
    \bigskip
    
    $V=\dfrac{P}{I}$&$R=\dfrac{V^{2}}         {P}$&$I=\dfrac{P}{V}$&$P=RI^{2}$\\ 
    
   \bigskip 
    
    $V=\sqrt{PR}$&$R=\dfrac{P}{I^{2}}$&$I=\sqrt{\dfrac{P}{R}}$&$P=\sqrt{\dfrac{V^{2}}{R}}$\\
    \hline
\end{tabular}
\end{center}
\label{tab:multicol}
\end{table}

$$
\underbrace{n(n-1)(n-2)\dots(n-m+1)}_
{\mbox{total of $m$ factors}}
$$

\bigskip  
  
\[\cos x=\sum_{k=0}^{\infty}\frac{(-1)^k}{(2k)!}x^{2k}\]

\bigskip 

\begin{center}
$A_{m,n} =
 \begin{pmatrix}
  a_{1,1} & a_{1,2} & \cdots & a_{1,n} \\
  a_{2,1} & a_{2,2} & \cdots & a_{2,n} \\
  \vdots  & \vdots  & \ddots & \vdots  \\
  a_{m,1} & a_{m,2} & \cdots & a_{m,n}
 \end{pmatrix}$
\end{center}

\subsection{Art ASCII}
\begin{verbatim}
               ,        ,
              /(        )`
              \ \___   / |
              /- _  `-/  '
             (/\/ \ \   /\
             / /   | `    \
             O O   ) /    |
             `-^--'`<     '
            (_.)  _  )   /
             `.___/`    /
               `-----' /
  <----.     __ / __   \
  <----|====O)))==) \) /====
  <----'    `--' `.__,' \
               |        |
                \       /
           ______( (_  / \______
         ,'  ,-----'   |        \
         `--{__________)        \/   
\end{verbatim}










\pagebreak











\subsection{Figure}
\begin{figure}[hbtp]
\caption{Figure listé en table des matières}
\centering
\fbox{\includegraphics[scale=0.7]{Figures/Bolid.jpg}}
\end{figure}

\subsection{Tableau}
\begin{table}[!ht]
\caption{Tableau listé en table des matières}
\medskip
\centering
\begin{tabular}{||l|l|l|l|l|l|l||}
\hline 
\textbf{Véhicule} & \textbf{Prix} & \textbf{L / 100 Km} & \textbf{Capacité} & \textbf{Comfort} & \textbf{Essence / année(\$)} & \textbf{Total} \\ 
\hline 
Ford Focus & 17 000 & 9.6 & 674-1209 & 5 & 12 480 & 24 660 \\ 
\hline 
Hyundai Elantra & 21 000 & 10.3 & 651-1449 & 4 & 13 390 & 27 570 \\ 
\hline 
Mazda 3 & 20 000 & 8.4 & 481-1213 & 4 & 10 920 & 24 600 \\ 
\hline 
Chevrolet Sonic & 17 000 & 7.7 & 459-946 & 5 & 10 010 & 22 190 \\ 
\hline 
Ford Fiesta & 14 000 & 9 & 504-1426 & 5 & 11 700 & 22 380 \\ 
\hline 
Nissan Versa & 13 500 & 9 & 228-728 & 4 & 11 700 & 22 130 \\ 
\hline 
Toyota Yaris  & 20 000 & 8.2 & 228-728 & 5 & 10 660 & 24 340 \\ 
\hline 
Honda Fit & 16 000 & 8.5 & 585-1622 & 5 & 11 050 & 22 730 \\ 
\hline 
\end{tabular} 
\label{tab:testtab1}
\end{table}

\pagebreak

\subsection{Multiples colones de texte}
\begin{multicols}{2}


Bacon ipsum dolor sit amet boudin ham hock turducken venison. Leberkas shank swine, drumstick flank ground round tenderloin jerky short loin cow venison boudin tri-tip chuck. Ham hock sausage ham, porchetta frankfurter rump t-bone jerky andouille pork chop pig shank ground round pork belly. Tongue rump ham jowl salami. Rump fatback turducken pork spare ribs landjaeger, chuck bresaola sirloin boudin. Porchetta spare ribs boudin ball tip pork.

Jerky andouille biltong pork belly drumstick. Jowl filet mignon tail chuck doner drumstick turkey. Rump flank strip steak shoulder pork chop shank. Flank short ribs drumstick ground round meatloaf.

Fatback turkey boudin bresaola, t-bone tail jowl leberkas. Tail jerky tenderloin ham hock pancetta, ribeye pork t-bone flank short ribs bacon cow. Turkey kevin t-bone shankle, jowl doner short loin ground round tri-tip rump sirloin. Beef tenderloin landjaeger boudin spare ribs pork belly andouille meatloaf salami corned beef flank. Kevin pancetta ribeye, chuck boudin tri-tip pork chop tongue. Leberkas porchetta sausage pork belly tri-tip pancetta ground round tail corned beef venison kevin ribeye.

Ground round pork ham pork chop cow, sausage andouille boudin. Rump andouille shank pork belly, chuck ribeye frankfurter boudin bresaola swine meatloaf tenderloin short loin beef ribs biltong. Beef tenderloin flank t-bone, shankle fatback prosciutto pork loin turducken turkey. Chicken fatback strip steak, drumstick biltong swine andouille rump porchetta pastrami turducken frankfurter t-bone tenderloin. Turducken pastrami pancetta short loin.

Pork chop ham hock tenderloin tail meatball meatloaf t-bone spare ribs. Boudin pancetta hamburger, drumstick ball tip t-bone shankle ribeye turkey. Shankle short loin turkey meatball t-bone. Tenderloin meatloaf kevin short ribs kielbasa sausage shank spare ribs pork chop cow chuck. Landjaeger ham hock rump, ham pork chop bresaola kielbasa turducken t-bone chuck tri-tip jerky boudin corned beef shoulder. Bresaola strip steak biltong ball tip turducken. Shankle ribeye pastrami meatball, tongue strip steak rump chuck beef ribs.
\end{multicols}
\begin{center}
\HRule 
\end{center}
\begin{multicols}{3}
Collaboratively administrate empowered markets via plug-and-play networks. Dynamically procrastinate B2C users after installed base benefits. Dramatically visualize customer directed convergence without revolutionary ROI.

Efficiently unleash cross-media information without cross-media value. Quickly maximize timely deliverables for real-time schemas. Dramatically maintain clicks-and-mortar solutions without functional solutions.

Completely synergize resource sucking relationships via premier niche markets. Professionally cultivate one-to-one customer service with robust ideas. Dynamically innovate resource-leveling customer service for state of the art customer service.

Objectively innovate empowered manufactured products whereas parallel platforms. Holisticly predominate extensible testing procedures for reliable supply chains. Dramatically engage top-line web services vis-a-vis cutting-edge deliverables.

Proactively envisioned multimedia based expertise and cross-media growth strategies. Seamlessly visualize quality intellectual capital without superior collaboration and idea-sharing. Holistically pontificate installed base portals after maintainable products.

Phosfluorescently engage worldwide methodologies with web-enabled technology. Interactively coordinate proactive e-commerce via process-centric "outside the box" thinking. Completely pursue scalable customer service through sustainable potentialities.

Collaboratively administrate turnkey channels whereas virtual e-tailers. Objectively seize scalable metrics whereas proactive e-services. Seamlessly empower fully researched growth strategies and interoperable internal or "organic" sources.

Credibly innovate granular internal or "organic" sources whereas high standards in web-readiness. Energistically scale future-proof core competencies vis-a-vis impactful experiences. Dramatically synthesize integrated schemas with optimal networks.Collaboratively administrate empowered markets via plug-and-play networks. Dynamically procrastinate B2C users after installed base benefits. Dramatically visualize customer directed convergence without revolutionary ROI.

Efficiently unleash cross-media information without cross-media value. Quickly maximize timely deliverables for real-time schemas. Dramatically maintain clicks-and-mortar solutions without functional solutions.

Completely synergize resource sucking relationships via premier niche markets. Professionally cultivate one-to-one customer service with robust ideas. Dynamically innovate resource-leveling customer service for state of the art customer service.

Objectively innovate empowered manufactured products whereas parallel platforms. Holisticly predominate extensible testing procedures for reliable supply chains. Dramatically engage top-line web services vis-a-vis cutting-edge deliverables.

Proactively envisioned multimedia based expertise and cross-media growth strategies. Seamlessly visualize quality intellectual capital without superior collaboration and idea-sharing. Holistically pontificate installed base portals after maintainable products.

Phosfluorescently engage worldwide methodologies with web-enabled technology. Interactively coordinate proactive e-commerce via process-centric "outside the box" thinking. Completely pursue scalable customer service through sustainable potentialities.

Collaboratively administrate turnkey channels whereas virtual e-tailers. Objectively seize scalable metrics whereas proactive e-services. Seamlessly empower fully researched growth strategies and interoperable internal or "organic" sources.

Credibly innovate granular internal or "organic" sources whereas high standards in web-readiness. Energistically scale future-proof core competencies vis-a-vis impactful experiences. Dramatically synthesize integrated schemas with optimal networks.

\end{multicols}

\section{Objectifs}
\begin{itemize}
\item[$\Rightarrow$]Vérification du bon fonctionnement du clavier
\item[$\Rightarrow$]Réutilisation des classes (C++) vues en session 4 pour lire les touches du clavier
\item[$\Rightarrow$]Vérification du bon fonctionnement de l’affichage
\item[$\Rightarrow$]Réutilisation des classes (C++) vues en session 4 pour affichage de caractères
\item[$\Rightarrow$]Vérification du bon fonctionnement du port série
\item[$\Rightarrow$]Réutilisation des  classes (C++) vues en session 4  pour transmission et réception de caractères sur le port UART
\item[$\Rightarrow$]Écrire un programme intégrant les trois modules :
\begin{itemize}
\item[•]Lecture d'un caractère du clavier par procédure d’interruption matériel 
\item[•]Transmission du même caractère sur l’UART ("9600, 8N1") 
\item[•]Lecture du port série par procédure d’interruption logicielle 
\item[•]Affichage sur LCD des caractères envoyés et reçus 
\end{itemize}
\item[$\Rightarrow$]Proposer une procédure pour la vérification de l’ensemble du programme
\item[$\Rightarrow$]Toutes autres améliorations que vous jugez pertinentes. Par exemple un simple menu pour configuration du port série
\end{itemize}


\section{Introduction}
Ce laboratoire a pour objectif de faire un retour sur les connaissances acquises en programmation au cours des dernières sessions. Entre autres, il faut écrire un programme pour la carte uPSD à l'aide d'IAR. Ce programme devra utiliser les interruptions afin lire lire un clavier matriciel et afficher les touches pesées sur un écran LCD tout en les envoyant dans le port série. De plus, le programme devra lire les caractères en provenance du port série pour les afficher sur l’écran.

\section{Installation et spécifications}

\subsection{Communication série}
Puisque ce laboratoire utilise abondamment les interruptions du port série et que la lecture des interruptions est au cœur du projet, nous en dressons ici la liste:


\begin{table}[!ht]
\centering
\caption{Registres d'initialisation de l'UART 1 uPSD}
\begin{tabular}{|l|l|l|}
\hline 
\textbf{Registre} & \textbf{Valeur} & \textbf{Description} \\ 
\hline 
IE\_bit.EA & = 0 & Désactive toutes les interruptions \\ 
\hline 
TMOD  & $\mid$= 0x20 & Timer 1, 8-bit auto-reload \\ 
\hline 
SCON0 & = 0x50 & UART1, mode 1, 1 start bit , 8 data bit, 1 stop bit  \\ 
\hline 
PCON & = PCON $\mid$ 0x80 & Doubleur de baud rate \\ 
\hline 
TH1 & = 0xF3 & Pour avoir 9600 bds sur l'uPSD \\ 
\hline 
ET1 & = 0 & Désactive les interruptions du Timer 1 \\ 
\hline 
TR1 & = 1 & Démarre le Timer 1 \\ 
\hline 
ES & = 1 & Active les interruptions du port serie\\
\hline
TI\_0 & = 0 &  Met a 0 le drapeau de la transmission série \\ 
\hline 
\end{tabular} 
\label{tab:testtab1}
\end{table}

\begin{table}[!ht]
\centering
\caption{Vecteurs d'interruption du uPSD}
\begin{tabular}{|l|l|}
\hline 
\textbf{Type d'interruption} & \textbf{Directive d'interruption}  \\ 
\hline 
Serial Port Interrupt & pragma vector = 0x23 \\ 
\hline 
Timer 0 Interrupt & pragma vector = 0x0B \\ 
\hline 
Timer 1 Interrupt & pragma vector = 0x1B \\ 
\hline 
External interrupt 0 & pragma vector = 0x03 \\ 
\hline 
External interrupt 1 & pragma vector = 0x13 \\ 
\hline 
\end{tabular} 

\label{tab:testtab1}
\end{table}

\pagebreak
\subsection{Lecture du clavier}
La lecture du clavier sur l'uPSD se fait à l'aide d'un 74HC922. Lorsqu'une touche du clavier matriciel est appuyée, le 74HC922 met un niveau logique UN sur la patte no. 14, ce qui a pour effet d'engendrer un Erternal Interrupt 0 sur l'uPSD. Lorsque ce dernier reçoit l'INTERRUPT 0, il active le ChipSelect 0 afin de lire et enregistrer sur son bus de données quatre valeurs binaires provenant du 74HC922. Les valeurs lues sont traitées afin de déterminer avec précision, antirebons inclus, la touche appuyée.

\section{Fonctionnement}
Lorsque l'utilisateur appuie sur une touche du clavier de la carte uPSD, la touche en question est affichée sur l'écran LCD du uPSD et est envoyée en RS-232 vers un ordinateur qui l'affiche à son tour à l'aide d'Hyperterm. Inversément, lorsque l'utilisateur appuie sur une touche du clavier de son ordinateur, la touche est envoyée par le port série en RS-232 puis est affichée sur l'écran LCD de la carte uPSD.






\subsection{Héritage de classe}
Les héritages de classes sont peu nombreux dans cet exercice. Essentiellement, les classes CLClavier et CLEcran héritent de la classe CLInOutBase qui sert à faire des accès matériels de bas niveau à des adresses spécifiques et spécifiés par les classes appelantes. Bref, les héritages sont tels qu'illustrés à la figure suivante:
\begin{figure}[hbtp]
\caption{Héritage de classe}
\centering
\fbox{\includegraphics[scale=1]{Figures/heritage7.png}}
\end{figure}




\fancyfoot[L]{Conclusion}
\section{Conclusion}
Nous avons eu quelques soucis avec la carte uPSD en ce qui a trait à la procédure de programmation. Bien qu'une défectuosité hardware de la carte soit à l'origine du problème, nous avons d'abord cru qu'il s'agissait d'un problème avec le RLINK JTAG, puis avec les drivers du RLINK JTAG, puis avec la machine virtuelle elle-même. Nous avons perdu du temps à isoler et à trouver le problème. Les autres difficultés concernent toutes des erreurs de compilations ayant pour origine de petits problèmes subtils, comme une parenthèse au mauvais endroit.






























\pagebreak
\fancyfoot[L]{ANNEXE}
\section{ANNEXE}
\subsection{Code source}
\begin{lstlisting}[label={list:first},caption=Code source]
// **************************CLClavier.cpp
// Auteur:      Vincent Chouinard
// Date:        17 avril 2014
// Version :    1.0
//
// Compilateur: IAR 8.1
//
// Description: Fichier permettant d'utiliser le clavier de l'uPSD
//
///////////////////////////////////////////////////////////////////////////////
//                     Infos relatives au montage du clavier sur le uPSD
// Utilise un 74HCC922
// 74HCC922     uPSD
//        A sur D0
//        B sur D1
//        C sur D2
//        D sur D3
//       OE sur CS1     (OutputEnable sur chip select 1)
//       DA sur INT0    (DataAvailible sur interrupt0)
///////////////////////////////////////////////////////////////////////////////
#include "CLClavier.h"            // Fichier de definitions du clavier

UC  CLClavier  :: ucEtatClavier = INACTIF;

//******************************************************************************
//                            LE CONSTRUCTEUR PAR DEFAUT
// *****************************************************************************
CLClavier :: CLClavier(void)
{
EX0 = 1;
IT0 = 1;
}

// *****************************************************************************
//                            LE DESTRUCTEUR
// *****************************************************************************
CLClavier :: ~CLClavier(void)
{

}

// ********************** FONCTION: ucBoutonAppuye()
//
// DESCRIPTION:         Fonction pour transmettre un caractere
//
// INCLUDE:             "Clavier.h"
//
// PROTOTYPE:           UC ucBoutonAppuye(void)
//
// PROCEDURE D'APPEL:   UC = ucBoutonAppuye()
//
// PARAMETRE D'ENTREE:  AUCUN
//
// PARAMETRE DE SORTIE: UC ---> Contient le bouton appuye
//
// Auteur:              Vincent Chouinard
// Date:                17 avril 2014 (Version 1.0)
// *****************************************************************************
UC CLClavier :: ucBoutonAppuye(void)
{
UC ucEtat;
if(ucEtatClavier == ACTIF)
  {
   ucEtat = ACTIF;
  }
else
  {
   ucEtat = INACTIF;
  }
return(ucEtat);
}

UC CLClavier:: ucLireClavier(void)
{
UC  uctTouche[]={         // Tableau des correspondances des
    '1','2','3','A',      // touches du clavier et de leurs
    '4','5','6','B',      // positions.
    '7','8','9','C',      //
    '*','0','#','D'};     //

ucToucheClavier = ucIn(0xF900);
ucEtatClavier   = INACTIF;
return(uctTouche[ucToucheClavier]);
}

// ********************** FONCTION: vkeyboardInterrupt()
//
// DESCRIPTION:         Fonction d'interruption pour lire le clavier
//
// INCLUDE:             "Clavier.h"
//
// PROTOTYPE:           __interrupt void vkeyboardInterrupt(void)
//
// PROCEDURE D'APPEL:   #pragma vector = 0x03 (Seul un interrupt0 peut appeler
//                                             cette fonction)
//
// PARAMETRE D'ENTREE:  AUCUN
//
// PARAMETRE DE SORTIE: AUCUN
//
// Auteur:              Vincent Chouinard
// Date:                17 avril 2014 (Version 1.0)
// *****************************************************************************
#pragma vector = 0x03 // Lors d'une interruption de type INTERRUPT 0
__interrupt void CLClavier :: vKeyboardInterrupt(void)
{
EA = 0;
ucEtatClavier = ACTIF;
EA = 1;
}
//LinuxLinuxLinuxLinuxLinuxLinuxLinuxLinuxLinuxLinuxLinuxLinuxLinuxLinuxLinuxTUX
//**********************  Fichier: CLCommunic.cpp
//  Description : Fonctions necessaire pour communiquer avec le port serie:
//
//
//CLCommunic()                                          Constructeur par defaut
//CLCommunic(UC ucLongBuf, UC ucLongTrame, UI uiBaud)  Construct initialisateur
//~CLCommunic()                                                     Destructeur
//UC ucInitLongTrame(UC ucLong)                 Pour initialiser longueur trame
//UC static ucLireNbCaractRecu()         Fct pour lire nombre de caractere recu
//UC ucLireEtatTrame()                         Fct pour lire l'etat de la trame
//char *ucpLireTrame()                Fct qui retourne adresse du debut lecture
//vTransCaratere(UC ucCar)                    Fct pour transmettre un caractere
//vTransChiffre(UI uiChiffreAAfficher)          Fct pour transmettre un chiffre
//vAffChaine(char const *ucpMessage)   Fct pour transmettre chaine de caractere
//void vInitBaudRate(UI uiBaud)        Fct pour intialiser le baudrate et timer
//__interrupt static void vInteruptSerial();     Fct pour recevoir un caractere
//UC ucValideTrame()                                 Fct pour valider une trame
//
//  Composition: aucune
//  Heritage : aucun
//
//  Programmeur: Philippe Dubois
//  Cours: 247-436
//
//  Date: 10 avril 2014
//                                                                           //
//  Compilateur: IAR 8.1
//                                                                           //
//  Modification:                                                            //
//                                                                           //
///////////////////////////////////////////////////////////////////////////////
#include "CLCommunic.h"             //inclue CLCommunic.h.
#include "ConversionKeilToIAR.h"

 ///// Initialise les variables statiques.
UC    CLCommunic :: ucEtatTrame       = INCOMPLET;      // Trame incomplete.
UC    CLCommunic :: ucNbCaractRecu    = 0;              // Nombre de caractere recu = 0
UC    CLCommunic :: ucLongueurTrame   = LONGUEURTRAME;  // Initialise les longueur du
UC    CLCommunic :: ucLongueurTampon  = LONGUEURTAMPON; // buffer et trame.
UC    CLCommunic :: ucCaractereRecu   = NULL;
UC    CLCommunic :: ucEtatReception   = INACTIF;
char *CLCommunic :: cpDebutTrame      = NULL;           // Initialise lse pointeur a NULL.
char *CLCommunic :: cpFinTrame        = NULL;
char *CLCommunic :: cptrEcriture      = NULL;
char *CLCommunic :: cptrLecture       = NULL;
char *CLCommunic :: cTabCaractereRecu = NULL;


//*****************************************************************************
// Nom de la fct:   CLCommunic(void)
// Description:     Constructeur qui permet d'initialiser par defaut le port
//                  serie du micro controleur a  9600 bdr, un tampon de 12
//                  et une trame a 4 characteres. Initialise les differents
//                  pointeurs au debut et la la fin du tampon.
//
// INCLUDE:         #include "_TypeCible.h", "CLCommunic.h"
// Prototype:       CLCommunic(void);
//
//
// Parametre d'entree: Aucune
//
// Parametre de sortie: Aucune
//
// Procedure appelees: vInitBaudRate(),
//
// Exemple d'appel: class CLCommunic clCommunic();
//
// Fait par: Philippe Dubois
// Date:            10 avril 2014
// Revision :       A
// Modification :
//*****************************************************************************
CLCommunic::CLCommunic(void)
{
 cTabCaractereRecu = new char[LONGUEURTAMPON]; // Definit tableau de longueur
                                               // LONGUEURTAMPON.
 ucLongueurTrame  = LONGUEURTRAME;             // Definit longueur de la trame.
 ucLongueurTampon = LONGUEURTAMPON;            // Definit longueur tampon.

 cpDebutTrame = &cTabCaractereRecu[0];                  // Pointeur sur debut tampon.
 cpFinTrame   = &cTabCaractereRecu[(LONGUEURTAMPON-1)]; // Pointeur sur fin tampon
 cptrEcriture = &cTabCaractereRecu[0];                  // Pointeur ecriture au debut.
 cptrLecture  = &cTabCaractereRecu[0];                  // Pointeur lecture au debut.

 vInitBaudRate(9600);                                   // Initialise le port serie.
}


//*****************************************************************************
// Nom de la fct:   CLCommunic
// Description:     Constructeur qui permet d'initialiser  le port
//                  serie du micro controleur a  la vitesse voulue, un tampon
//                  de taille dynamique et la longueur de la trame.
//                  Initialise les differents pointeurs au debut et la la fin
//                  du tampon.
//
// INCLUDE:         #include "_TypeCible.h", "CLCommunic.h"
// Prototype:     CLCommunic(UC ucLongBuf, UC ucLongTrame, UI uiBaud);
//
//
// Parametre d'entree: ucLongBuf pour longueur du trableau
//                     ucLongTrame pour longueur trame
//                     uiBaud pour la vitesse du port serie
//
// Parametre de sortie: Aucune
//
// Procedure appelees: vInitBaudRate(), ucInitLongTrame()
//
// Exemple d'appel: class CLCommunic clCommunic();
//
// Fait par: Philippe Dubois
// Date:            10 avril 2014
// Revision :       A
// Modification :
//*****************************************************************************
CLCommunic :: CLCommunic(UC ucLongBuf, UC ucLongTrame, UI uiBaud)
{
 cTabCaractereRecu = new char[ucLongBuf]; // Initialise un tableau de longueur
                                          // voulue.
 ucLongueurTampon = ucLongBuf;            // Longueur tampon = ucLongBuf.
 ucInitLongTrame(ucLongTrame);            // Verifie si longueur trame
                                          // < longueur tampon.
 cpDebutTrame = &cTabCaractereRecu[0];    // Pointeur pour debut tampon.
 cpFinTrame   = &cTabCaractereRecu[(ucLongBuf-1)];// Pointeur pour fin tampon.
 cptrEcriture = &cTabCaractereRecu[0];    // Pointeur ecriture et lecture au
 cptrLecture  = &cTabCaractereRecu[0];    // debut du tampon.
 vInitBaudRate(uiBaud);                   // Initialise la vitesse de
                                          // communication.
}

//*****************************************************************************
// Nom de la fct:   ~CLCommunic
// Description:     Desctructeur qui libert la memoire pris par le tampon.
//
// INCLUDE:        #include "_TypeCible.h", "CLCommunic.h"
// Prototype:      ~CLCommunic(void);
//
//
// Parametre d'entree: aucun
//
// Parametre de sortie: Aucun
//
// Procedure appelees: aucun
//
// Exemple d'appel:
//
// Fait par: Philippe Dubois
// Date:            10 avril 2014
// Revision :       A
// Modification :
//*****************************************************************************
CLCommunic :: ~CLCommunic(void)
{
if(cTabCaractereRecu)            // Verification pour s'assurer que le
  {                                // pointeur n'a pas deja ete libere
   delete cTabCaractereRecu;       // Libert la memoire prise par le tampon.
   cTabCaractereRecu = NULL;       // Pour mettre a 0 le pointeur.
  }
}

//*****************************************************************************
// Nom de la fct:   ucInitLongTrame
// Description:     Verifie si la trame entree est inferieure au tampon
//                  sinon la longueur de la trame = a celle du buffer.
//
// INCLUDE:        #include "_TypeCible.h", "CLCommunic.h"
// Prototype:      UC ucInitLongTrame(UC ucLong);
//
//
// Parametre d'entree: ucLong longueur voulue
//
// Parametre de sortie: ucValide pour valide ou non la taille
//
// Procedure appelees: aucun
//
// Exemple d'appel:  ucInitLongTrame(ucLongTrame);
//
// Fait par: Philippe Dubois
// Date:            10 avril 2014
// Revision :       A
// Modification :
//*****************************************************************************

UC CLCommunic :: ucInitLongTrame(UC ucLong)
{
UC ucValide;                           // Varaible pour valider ou non.

if(ucLong <= ucLongueurTampon)         // Si longueur trame voulue <= a la longueur
  {                                    // du tampon.
   ucLongueurTrame = ucLong;           // longueur trame = longueur trame voulue.
   ucValide = VRAI;                    // ucValide = Vrai.
  }

if(ucLong > ucLongueurTampon)          // Si longueur trame voulue > a la longueur
  {                                    // du tampon.
   ucLongueurTrame = ucLongueurTampon; // longueur trame = longueur tampon.
   ucValide = FAUX;                    // ucValide = Faux.
  }

return(ucValide);                      // Retourne etat de ucValide.
}


//*****************************************************************************
// Nom de la fct:   ucValideTrame
// Description:     Valide la trame recue, additionne tous les caracteres
//                  recues et commpare avec le checksum.
//
// INCLUDE:        #include "_TypeCible.h", "CLCommunic.h"
// Prototype:      UC ucValideTrame(void);
//
//
// Parametre d'entre: aucun
//
// Parametre de sortie: ucValide pour valide ou non la trame.
//
// Procedure appelees: aucun
//
// Exemple d'appel: if(ucValideTrame() == VRAI)
//
// Fait par: Philippe Dubois
// Date:            10 avril 2014
// Revision :       A
// Modification :
//*****************************************************************************
UC CLCommunic :: ucValideTrame(void)
{
UC ucValide;               // Variable pour etat valide ou non.
UC ucSomme;                // variable pour la somme des caracteres recus.
ucSomme = 0;               // Initialise la somme a 0.
UC ucBoucle;               // Varialbe pour boucle.

char *cPtrVerification;    // pointeur sur la tramme recu
cPtrVerification = cptrLecture; // Intialise l'adresse du debut de la trame.

ucBoucle = (ucLongueurTrame - 1);         // Pour additionner toutes les valeurs
                                          // sauf le dernier
for(;ucBoucle > 0; ucBoucle--)
  {
   ucSomme = *cPtrVerification + ucSomme; // Additionne le caractere avec somme.
   cPtrVerification ++;                   // Valeur suivante.
  }
                                          // Compare la somme avec le checksum recu.
if(ucSomme == *cPtrVerification)          // Si egal?
  {
   ucValide = VRAI;                       // Trame correcte.
  }
else                                      // Sinon
  {
   ucValide = FAUX;                       // Trame incorrecte.
  }

return(ucValide);                         // Retourne valide ou non.
}

//*****************************************************************************
// Nom de la fct:   vTransCaratere                  Fait par: Philippe Dubois
// Date:            2 octobre 2013
// Revision :       A
// Modification :
// INCLUDE:         #include "_TypeCible.h", "CLCommunic.h"
// Prototype:       void vTransCaratere(UC ucCar);
//
// Description:     Fonction qui permet de transmettre un caractere par le port
//                  serie, et on attend que le caractere soit completement
//                  envoye par "polling" avec le flag TI_0.
//
// Parametre d'entree: ucCar: caractere qu'on veut envoyer.
//
//
// Parametre de sortie: aucune
//
// Procedure appelees: aucune.
//
// Exemple d'appel: clCommunic.vTransCaratere('A')
//*****************************************************************************
void CLCommunic :: vTransCaratere(UC ucCar)
{
SBUF0 = ucCar;             // Envoye le caractere dans SBUF0 pour le transmetre sur TX
while(TI_0 ==0);           // Si le flag TI_0 se met a 1 (termine)
TI_0  = 0;                 // Remet le flag a 0.
}



//*****************************************************************************
// Nom de la fct:   vAffChaine
// Description:     Transmet une chaine de caracteres sur le port serie.
//
// INCLUDE:        #include "_TypeCible.h", "CLCommunic.h"
// Prototype:      void vAffChaine(char const *ucpMessage)
//
//
// Parametre d'entre: *ucpMessage pour chaine de caractere.
//
// Parametre de sortie: aucun
//
// Procedure appelees: vTransCaratere()
//
// Exemple d'appel: clCommunic.vAffChaine("TEST");
//
// Fait par: Philippe Dubois
// Date:            10 avril 2014
// Revision :       A
// Modification :
//*****************************************************************************
void CLCommunic :: vAffChaine(char const *ucpMessage)
{
while(*ucpMessage != 0x00)          // Tant qu'on a pas atteint la fin de
  {                                 // la chaine.
   vTransCaratere(*ucpMessage);     // Transmet le caractere.
   ucpMessage++;                    // Passe au caractere suivant.
  }
}

//*****************************************************************************
// Nom de la fct:   ucLireNbCaractRecu                Fait par: Philippe Dubois
// Date:            2 octobre 2013
// Revision :       A
// Modification :
// INCLUDE:         #include "_TypeCible.h", "CLCommunic.h"
// Prototype:       UC ucLireNbCaractRecu(void);
//
// Description:     Fonction qui retourne le nombre de caractere recu.
//
//
// Parametre d'entree: aucun
//
//
// Parametre de sortie: ucNombreCaractere: nombre de caractere recu.
//
// Procedure appelees: aucune.
//
// Exemple d'appel: if(ucLireNbCaractRecu() < (ucLongueurTrame -1))
//*****************************************************************************
UC CLCommunic :: ucLireNbCaractRecu(void)
{
return(ucNbCaractRecu);          // Retourne le nombre de caractere recu.
}
//*****************************************************************************
// Nom de la fct:   ucLireEtatTrame                  Fait par: Philippe Dubois
// Date:            2 octobre 2013
// Revision :       A
// Modification :
// INCLUDE:        #include "_TypeCible.h", "CLCommunic.h"
// Prototype:       UC ucLireEtatTrame(void);
//
// Description:     Fonction qui permet de savoir si une trame est complete.
//
//
// Parametre d'entree: aucun
//
//
// Parametre de sortie: ucEtatTrameRecu: a 1 pour complete et 0 si imcomplete.
//
// Procedure appelees: aucune.
//
// Exemple d'appel:if(ucLireEtatTrame()==TRAMECOMPLET)
//*****************************************************************************
UC CLCommunic :: ucLireEtatTrame(void)
{
return(ucEtatTrame);            // Retourne l'etat de la trame.
}
//*****************************************************************************
// Nom de la fct:   *ucpLireTrame                   Fait par: Philippe Dubois
// Date:            2 octobre 2013
// Revision :       A
// Modification :
// INCLUDE:         #include "_TypeCible.h", "CLCommunic.h"
// Prototype:       UI *ucpLireTrame(void);
//
// Description:     Fonction qui permet de transmettre l'adresse du tableau
//                  contenant la trame de caractere, et reinitialiser l'etat
//                  de la  trame.
//
// Parametre d'entree: aucun
//
//
// Parametre de sortie: uiTab: adresse du tableau.
//
// Procedure appelees: ucValideTrame()
//
// Exemple d'appel:vTransCaratere('A')
//*****************************************************************************
char *CLCommunic :: ucpLireTrame(void)
{
char *ucptrLectAEnvoyer;           // Pointeur sur la trame recue.
ucEtatTrame = INCOMPLET;           // Reinitialise l'etat de la trame.

if(ucValideTrame() == VRAI)        // Verifie si la trame est valide.
  {                                // Si elle l'est
   ucptrLectAEnvoyer = cptrLecture;// transmet l'adresse du debut de la trame.
  }
else                               // Sinon
  {
   ucptrLectAEnvoyer = NULL;       // Ne transmet pas l'adresse du debut de
  }                                // la trame.
return(ucptrLectAEnvoyer);         // Retourne l'adresse du tableau ou NULL.
}
//*****************************************************************************
// Nom de la fct:   vInitBaudRate                  Fait par: Philippe Dubois
// Date:            2 octobre 2013         version base
//                  10 avril 2014          version avec uPSD.
// Revision :       B
// Modification :
// INCLUDE:         #include "_TypeCible.h", "CLCommunic.h"
// Prototype:       void vInitBaudRate(UI uiBaud);
//
// Description:     Fonction qui permet d' initialiser le port série au
//                  baudrate desire, le timer 1(8 bit auto reload) et
//                  les interuptions necessaire pour communiquer avec
//                  le port serie.
//                  Ajout de la compilation conditionnelle pour choix
//                  de la valeur a mettre dans TH1 selon la cible.
//
// Parametre d'entree: UI uiBaud: pour determine la vitesse du baudrate.
//
//
// Parametre de sortie: aucune
//
// Procedure appelees: aucune.
//
// Exemple d'appel: vInitBaudRate(57600);
//*****************************************************************************
void CLCommunic:: vInitBaudRate(UI uiBaud)
{
SCON0 = 0x50;         // SM0_0=0
                      // SM1_0=1
                      // SM2_0=0 pour longeur 10 bits et asychrone.
                      // REN_0=1 pour recevoir des caracteres.
TMOD = TMOD | 0x20;   // M1=1
                      // M0=0 pour timer 8bit avec autoreload.
PCON = PCON | 0x80;   // Pour doubler le baudrate (SMOD_0=1;).

#ifdef DALLAS89C450   // Si la cible est le Dallas

switch(uiBaud)         // switch case pour determiner la valeur en
  {                    // hexa qui faut mettre a TH1 pour avoir la
                   // bonne vitesse du timer pour generer
   case 57600:         // le baudrate.
      TH1= 0xFF;
   break;
   case 38400:
      TH1= 0xFE;
   break;
   case 19200:
      TH1= 0xFD;
   break;
   case 9600:
      TH1= 0xFA;
   break;
   case 4800:
      TH1= 0xF4;
   break;
   case 2400:
      TH1= 0xE8;
   break;
   default:
      TH1= 0xFA;        // Si aucune valeur ne correspond vitesse = 9600 .
   break;
  }
#endif

#ifdef UPSD3254A        // Si la cible est le uPSD

switch(uiBaud)         // switch case pour determiner la valeur en
  {                    // hexa qui faut mettre a TH1 pour avoir la
                  // bonne vitesse du timer pour generer
    /*case 57600:      // le baudrate.
    TH1= 0xFE ;
    break;
    case 38400:        // uPSD ne peut aller a ces vitesses
    TH1= 0xFD;
    break;
    case 19200:
    TH1= 0xF9;
    break;*/
    case 9600:
       TH1= 0xF3;
    break;
    case 4800:
       TH1= 0xE6;
    break;
    case 2400:
       TH1= 0xCC;
    break;
    default:
       TH1= 0xF3;         // Si aucune valeur ne correspond vitesse = 9600 .
    break;
  }
#endif

TR1  = 1;      // Active le timer 1.
ET1  = 0;      // Desactive les interruptions du timer 1.
EA   = 1;      // Active les interruptions.
ES   = 1;      // Active les interruptions du port serie.
TI_0 = 0;      // Met a 0 le drapeau de la transmision serie.
}

//*****************************************************************************
// Nom de la fct:   vInteruptSerial                  Fait par: Philippe Dubois
// Date:            2 octobre 2013
//
// Revision :       B
// Modification :   10 avril 2014     conversion pour IAR
//
// INCLUDE:         #include "_TypeCible.h", "CLCommunic.h"
// Prototype:       __interrupt  void CLCommunic:: vInteruptSerial(void);
//
// Description:     Fonction qui permet de recevoir des caracteres par le port
//                  serie, si l'interruption vient de la reception on place
//                  les caracteres lus dans un tableau.
//
// Parametre d'entree: aucun.
//
//
// Parametre de sortie: valeur dans cptrLecture pour lire les caractere recus.
//
// Procedure appelees: ucLireNbCaractRecu(),
//
// Exemple d'appel: aucun
//*****************************************************************************
#pragma vector=0x23  // Interrupt 4 pour interruption serie au vecteur 23
 __interrupt  void CLCommunic:: vInteruptSerial(void)
{
EA = 0;                               // Desactive les interruptions.

if(RI_0==1)                           // Interruption cause par la reception?
  {
   if(cptrEcriture > cpFinTrame)      // Si on a atteind la fin du buffer
     {
      cptrEcriture = cpDebutTrame;    // Retourne au debut du tampon.
     }

  *cptrEcriture    = SBUF0;  // Lit dans SBUF0 le caractere recu et met dans le tableau.
   ucCaractereRecu = SBUF0;
   ucEtatReception = ACTIF;
   cptrEcriture++;           // Increment la position du pointeur d'ecriture.

   if(ucLireNbCaractRecu() < (ucLongueurTrame - 1) )
     {                       // Si le nombre de caractere recu est inferieur a
      ucNbCaractRecu++;      // trame - 1 incremente le nombre de caracteres
     }                       // recus.
   else                      // Si la trame est complete?
     {
      ucNbCaractRecu=0;      // Remet a 0 le compte de caractere.
      ucEtatTrame = COMPLET; // Indique qu'une trame est complete.
      cptrLecture = (cptrEcriture - (ucLongueurTrame));
                             // Le pointeur de lecture = la position
                             // d'ecriture - la longuer de la trame.
     }
   RI_0 = 0;                 // Reinitialiser le flag de reception.
 }                           // Fin du if(RI_0 = 1)
EA = 1;                      // Reactive les interruptions
}                            // Fin de Interruption.

//*****************************************************************************
// Nom de la fct:   vTransChiffre
// Description:     Fonction qui permet transmettre  un nombre de 0 a
//                  65000 sur le port serie.
// INCLUDE:       #include "_TypeCible.h", "CLCommunic.h"
// Prototype:     void vTransChiffre(UI uiChiffreAAfficher);
//
//
// Parametre d'entree: uiChiffreAAfficher pour chiffre a afficher
//
// Parametre de sortie: aucun
//
// Procedure appelees: vTransCaratere();
//
//
// Exemple d'appel: vLcdAffChiffre(12345)
//
// Fait par: Philippe Dubois
// Date:            14 novembre 2013
// Revision :       A
// Modification :
//*****************************************************************************
void CLCommunic :: vTransChiffre(UI uiChiffreAAfficher)
{
UC ucChiffre0; // Declaration des variables
UC ucChiffre1;
UC ucChiffre2;
UC ucChiffre3;
UC ucChiffre4; // isole chaque chiffre du nombre.

ucChiffre0 = ((uiChiffreAAfficher  % 10)    + 0x30);       // Unitee.
ucChiffre1 = (((uiChiffreAAfficher / 10)    % 10) + 0x30); // Dizaine
ucChiffre2 = (((uiChiffreAAfficher / 100)   % 10) + 0x30); // Centaine.
ucChiffre3 = (((uiChiffreAAfficher / 1000)  % 10) + 0x30); // Millier.
ucChiffre4 = ((uiChiffreAAfficher  / 10000) + 0x30);       // Dizaine de mille.

if(ucChiffre4 != 0x30)                          // Si Dizaine de mille  = 0
  {                                           // Ne l'affiche pas
   vTransCaratere(ucChiffre4);                  // Transmet les Dizaine de mille.
  }

if((ucChiffre4 != 0x30) || (ucChiffre3 != 0x30))// Si les 2 premiers chiffres
  {                                             // sont different de 0.
   vTransCaratere(ucChiffre3);                  // Transmet les millier.
  }

if((ucChiffre4 != 0x30) || (ucChiffre3 != 0x30)||(ucChiffre2 != 0x30))
  {                                  // Si les 3 1er chiffres sont different
                                       // de 0?
   vTransCaratere(ucChiffre2);         // Transmet les centaines.
  }

if((ucChiffre4 != 0x30)||(ucChiffre3 != 0x30)||(ucChiffre2 != 0x30)||(ucChiffre1 != 0x30))
                                  // Si les 4 premier chiffres sont
  {                                    // different de 0.
   vTransCaratere(ucChiffre1);         // Transmet les dizianes.
  }
 vTransCaratere(ucChiffre0);           // Transmetles unitees.
}

UC CLCommunic :: ucLireTrameSansCheck(void)
{
ucEtatReception = INACTIF;
return(ucCaractereRecu);
}

UC CLCommunic :: ucLireEtatReception(void)
{
return(ucEtatReception);            // Retourne l'etat de la trame.
}


//LinuxLinuxLinuxLinuxLinuxLinuxLinuxLinuxLinuxLinuxLinuxLinuxLinuxLinuxLinuxTUX






// **************************CLEcran.cpp
// Auteur:      Vincent Chouinard
// Date:        13 fevrier 2014
// Version :    1.0
//
// vLcdAffCar      ---> Affiche un caracter (a-z,A-Z0-9)
// vLcdClr         ---> Efface l'ecran
// vLcdPoscurC0L1  ---> Positionne le curseur
// vLcdInit        ---> Initialise l'ecran
// vDelaisEcran    ---> Delais a tout faire
// vLcdAffChaine   ---> Affiche un string de "Texte"
// vLcdAffEcran    ---> Pour afficher des menus
// vLcdAffCarHex   ---> Pour afficher un caracter en HEX
// vLcdCarGen      ---> Pour creer des caracteres customs
// vHexToAscii2    ---> Pour transformer l'ASCII en HEX
// vLcdBusy        ---> Busy Polling
//
// Compilateur: IAR 8.2 && Keil 4.79.9
//
// Description: Fichier de definitions pour utiliser un ecran LCD
//
// *****************************************************************************
#include "CLEcran.h"
// *****************************************************************************
//                            LES CONSTRUCTEURS
// *****************************************************************************
CLEcran :: CLEcran(void)
{
uiAdresseEcran = 0xF800;
vLCDInit();
}

CLEcran :: CLEcran(USI uiAdresse)
{
uiAdresseEcran = uiAdresse;
vLCDInit();
}

// *****************************************************************************
//                            LE DESTRUCTEUR
// *****************************************************************************
CLEcran :: ~CLEcran(void)
{
vLCDClr();
}

// **********************FONCTION: NomDeLaFonction()**************************
//
// DESCRIPTION: Positionne le curseur d'ecriture a l'endroit spécifie par
//              la fonction appelante
//
// INCLUDE: <_DeclarationGenerale.h>
//          "CLEcran.h"
//
// PROTOTYPE:           void vLcdPoscurC0L1(UC,UC)
//
// PROCEDURE D'APPEL:   vLcdPoscurC0L1(ucCol,ucLigne)
//
// PARAMETRE D'ENTREE:  ucCol   ---> Positionne la colone
//                      ucLigne ---> Positionne la ligne
//
// PARAMETRE DE SORTIE: AUCUN
//
// EXEMPLE: vLcdPoscurC0L1(0,1) ---> Met le curseur au debut de l'ecran
//
// Auteur:              Vincent Chouinard
// Date:                22 mars 2014   (Version 1.0)
// Modification:
// *****************************************************************************
void CLEcran :: vDelaiLCD(USI uiDelai)
{
USI i;
for(i = 0;i <= uiDelai; i++);
}

void CLEcran :: vLCDBusy(void)
{
while(ucIn(uiAdresseEcran + ECRANFLAG) & 0x80);
}

void CLEcran :: vLCDClr(void)
{
vLCDBusy();
vOut(uiAdresseEcran + ECRANCONFIG, 0x01);
}

void CLEcran :: vLCDInit(void)
{
USI i;
for(i = 0; i <= 3; i++)    //Faire 3 fois
  {
   vDelaiLCD(10000);                  //Attendre la stabilisation des parametres
   vOut(uiAdresseEcran + ECRANCONFIG,0x38);
  }

vLCDBusy();vOut(uiAdresseEcran + ECRANCONFIG,0x38); // Attente du Busy Flag
vLCDBusy();vOut(uiAdresseEcran + ECRANCONFIG,0x01); // Attente du Busy Flag
vLCDBusy();vOut(uiAdresseEcran + ECRANCONFIG,0x0C); // Attente du Busy Flag
vLCDBusy();vOut(uiAdresseEcran + ECRANCONFIG,0x06); // Attente du Busy Flag
}

// **********************FONCTION: NomDeLaFonction()**************************
//
// DESCRIPTION: Positionne le curseur d'ecriture a l'endroit spécifie par
//              la fonction appelante
//
// INCLUDE: <_DeclarationGenerale.h>
//          "CLEcran.h"
//
// PROTOTYPE:           void vLcdPoscurC0L1(UC,UC)
//
// PROCEDURE D'APPEL:   vLcdPoscurC0L1(ucCol,ucLigne)
//
// PARAMETRE D'ENTREE:  ucCol   ---> Positionne la colone
//                      ucLigne ---> Positionne la ligne
//
// PARAMETRE DE SORTIE: AUCUN
//
// EXEMPLE: vLcdPoscurC0L1(0,1) ---> Met le curseur au debut de l'ecran
//
// Auteur:              Vincent Chouinard
// Date:                22 mars 2014   (Version 1.0)
// Modification:
// *****************************************************************************
void CLEcran :: vLCDCursor(UC ucCol, UC ucLine)
{
UC ucPosition;
switch(ucLine)
  {
   case 1:                       //Si ligne 1 choisi
      ucPosition = 0x00 + ucCol; //Curseur sur ligne 1 (1,Y)
   break;

   case 2:                       //Si ligne 2 choisi
      ucPosition = 0x40 + ucCol; //Curseur sur ligne 2 (2,Y)
   break;

   case 3:                       //Si ligne 3 choisi
      ucPosition = 0x14 + ucCol; //Curseur sur ligne 3 (3,Y)
   break;

   case 4:                       //Si ligne 4 choisi
      ucPosition = 0x54 + ucCol; //Curseur sur ligne 4 (4,Y)
   break;

   default:                      //Sinon
      ucPosition = 0x00 + ucCol; //Curseur à position initial (0,0)
   break;
  }

ucPosition = ucPosition | 0x80; //Definition du registre du curseur
vLCDBusy();                                   //Attente du Flag Busy
vOut(uiAdresseEcran+ECRANCONFIG, ucPosition); //Envoi de la commande
}

void CLEcran :: vLCDDisplayCarac(UC ucCaractere)
{
vLCDBusy();
vOut(uiAdresseEcran+ECRANWR, ucCaractere);
}

void CLEcran :: vLCDDisplayCarac(char cCaractere)
{
vLCDBusy();
vOut(uiAdresseEcran+ECRANWR, cCaractere);
}

void CLEcran :: vHexToASCII(UC ucHexCar, UC *ucpHi, UC *ucpLo)
{
UC ucTemp;
ucTemp = ucHexCar;
*ucpHi = ucTemp   >> 4;
*ucpLo = ucHexCar & 0x0F;

if(*ucpHi <= 9)             //SI MSB est chiffe
  {
   *ucpHi = *ucpHi + 0x30;  //Convertion hexa a ASCII (Chiffre)
  }
else                        //SINON
  {
   *ucpHi = *ucpHi + 0x37;  //Convertion hexa a ASCII (Lettre)
  }

if(*ucpLo <= 9)             //SI LSB est chiffre
  {
   *ucpLo = *ucpLo + 0x30;  //Convertion hexa a ASCII (Chiffre)
  }
else                        //SINON
  {
   *ucpLo = *ucpLo + 0x37;  //Convertion hexa a ASCII (Lettre)
  }
}

void CLEcran :: vHexToASCII(USI uiHexCar, UC *ucpHiOctHi, UC *ucpLoOctHi,
                                          UC *ucpHiOctLo, UC *ucpLoOctLo)
{
*ucpHiOctHi = (uiHexCar >> 12) & 0x0F;
*ucpLoOctHi = (uiHexCar >> 8 ) & 0x0F;
*ucpHiOctLo = (uiHexCar >> 4 ) & 0x0F;
*ucpLoOctLo =  uiHexCar        & 0x0F;

if(*ucpHiOctHi <= 9)                  //SI MSB est chiffe
  {
   *ucpHiOctHi = *ucpHiOctHi + 0x30;  //Convertion hexa a ASCII (Chiffre)
  }
else                                  //SINON
  {
   *ucpHiOctHi = *ucpHiOctHi + 0x37;  //Convertion hexa a ASCII (Lettre)
  }

if(*ucpLoOctHi <= 9)                  //SI LSB est chiffre
  {
   *ucpLoOctHi = *ucpLoOctHi + 0x30;  //Convertion hexa a ASCII (Chiffre)
  }
else                                  //SINON
  {
   *ucpLoOctHi = *ucpLoOctHi + 0x37;  //Convertion hexa a ASCII (Lettre)
  }

if(*ucpHiOctLo <= 9 )                 //SI MSB est chiffre
  {
   *ucpHiOctLo = *ucpHiOctLo + 0x30;  //Convertion hexa a ASCII (Chiffre)
  }
else                                  //SINON
  {
   *ucpHiOctLo = *ucpHiOctLo + 0x37;  //Convertion hexa a ASCII (Lettre)
  }

if(*ucpLoOctLo <= 9 )                 //SI LSB est chiffre
  {
   *ucpLoOctLo = *ucpLoOctLo + 0x30;  //Convertion hexa a ASCII (Chiffre)
  }
else                                  //SINON
  {
   *ucpLoOctLo = *ucpLoOctLo + 0x37;  //Convertion hexa a ASCII (Lettre)
  }
}

void CLEcran :: vLCDDisplayHexCarac(UC ucHexCar)
 {
  UC ucHi;UC ucLo;

  vHexToASCII(ucHexCar, &ucHi, &ucLo); //Conversion de Hexadecimal a ASCII
  vLCDDisplayCarac(ucHi);              //Affiche MSB
  vLCDDisplayCarac(ucLo);              //Affiche LSB
 }

void CLEcran :: vLCDDisplayHexCarac(USI uiHexCar)
{
UC ucHiOctHi;
UC ucLoOctHi;
UC ucHiOctLo;
UC ucLoOctLo;

vHexToASCII(uiHexCar, &ucHiOctHi, &ucLoOctHi, &ucHiOctLo, &ucLoOctLo);

vLCDDisplayCarac(ucHiOctHi); //Affiche MSB OctHI
vLCDDisplayCarac(ucLoOctHi); //Affiche LSB OctHI
vLCDDisplayCarac(ucHiOctLo); //Affiche MSB OctLO
vLCDDisplayCarac(ucLoOctLo); //Affiche LSB OctLO
}

void CLEcran :: vLCDDisplayCaracChain(const UC *ucpMessage)
{
while(*ucpMessage != 0x00)
  {
   vLCDDisplayCarac(*ucpMessage);
   ucpMessage++;
  }
}

void CLEcran ::vLCDDisplayCaracChain (const char *cpMessage)
{
while(*cpMessage != 0x00)
  {
   vLCDDisplayCarac(*cpMessage);
   cpMessage++;
  }
}

void CLEcran :: vLCDDisplayScreen(const UC *ucpEcran)
{
UC ucLigne;
for(ucLigne = 1; ucLigne <= 4; ucLigne++)
  {
   vLCDCursor(1,ucLigne);
   vLCDDisplayCaracChain(ucpEcran);
   ucpEcran = ucpEcran + 1;
  }
}

void CLEcran :: vLCDDisplayEtatPort(UC ucEtatPort)
{
for(UC i = 0; i < 8; i++)
  {
   if((ucEtatPort & 0x80) == 0x80)
     {
      vLCDDisplayCarac('1');
     }
   else
     {
      vLCDDisplayCarac('0');
     }
   ucEtatPort = ucEtatPort << 1;
  }
}

void CLEcran :: vLCDDisplayFloat(float fInputFloat)
{
int i                 = 0; // Pour les boucles de comptage
float fTemporaire     = 0; // Pour le calcul de decimales
USI usiLesEntiers     = 0; // Pour la conversion d'un float en entier
USI usiLesEntiersTemp = 0; // Pour la conversion d'un float en entier
USI usiDecimalesTemp  = 0; // Pour convertir les decimales d'un float en entier
USI usiLesDecimales   = 0; // Pour convertir les decimales d'un float en entier

usiLesEntiers   = (USI)fInputFloat; // Conversion des floats en USI sans virgule
usiLesDecimales = (USI)fInputFloat; // Enleve les decimales du float
fTemporaire     = (fInputFloat - (float)usiLesDecimales) * 100;//Decimales 0-999
usiLesDecimales = (USI)fTemporaire; //Conversion (.0 a .999) vers (0 a 999)

for(i = 0; i < 3; i++) //Pour trois fois (Centaine, dizaine, unite)
  {
   if(i == 0) // Pour les centaines
     {
      usiLesEntiersTemp = usiLesEntiers /  100;
      usiLesEntiers     = usiLesEntiers - (100 * usiLesEntiersTemp);
     }
   if(i == 1) // Pour les dizaines
     {
      usiLesEntiersTemp = usiLesEntiers /  10;
      usiLesEntiers     = usiLesEntiers - (10 * usiLesEntiersTemp);
     }
   if(i == 2) // Pour les unitees
     {
      usiLesEntiersTemp=usiLesEntiers;
     }

   switch(usiLesEntiersTemp) // Affiche un caractere correspondant a l'unite
     {                       // float lue
      case 0:
         vLCDDisplayCarac('0');
      break;

      case 1:
         vLCDDisplayCarac('1');
      break;

      case 2:
         vLCDDisplayCarac('2');
      break;

      case 3:
         vLCDDisplayCarac('3');
      break;

      case 4:
         vLCDDisplayCarac('4');
      break;

      case 5:
         vLCDDisplayCarac('5');
      break;

      case 6:
         vLCDDisplayCarac('6');
      break;

      case 7:
         vLCDDisplayCarac('7');
      break;

      case 8:
         vLCDDisplayCarac('8');
      break;

      case 9:
         vLCDDisplayCarac('9');
      break;
     }
   }

vLCDDisplayCarac('.'); // Affiche la satane virgule

for(i = 0; i < 2; i++) //Pour deux fois (Dixieme, centieme)
  {
   if(i == 0)          // Pour les dixiemes
     {
      usiDecimalesTemp = usiLesDecimales /  10;
     }
   if(i == 1)          // Pour les centiemes
     {
      usiDecimalesTemp = usiLesDecimales - (10 * usiDecimalesTemp) ;
     }

   switch(usiDecimalesTemp) // Affiche un caractere correspondant a l'unite
     {                      // float lue
      case 0:
         vLCDDisplayCarac('0');
      break;

      case 1:
         vLCDDisplayCarac('1');
      break;

      case 2:
         vLCDDisplayCarac('2');
      break;

      case 3:
         vLCDDisplayCarac('3');
      break;

      case 4:
         vLCDDisplayCarac('4');
      break;

      case 5:
         vLCDDisplayCarac('5');
      break;

      case 6:
         vLCDDisplayCarac('6');
      break;

      case 7:
         vLCDDisplayCarac('7');
      break;

      case 8:
         vLCDDisplayCarac('8');
      break;

      case 9:
         vLCDDisplayCarac('9');
      break;
     }
  }
}
//LinuxLinuxLinuxLinuxLinuxLinuxLinuxLinuxLinuxLinuxLinuxLinuxLinuxLinuxLinuxTUX
// ************************** FICHIER: CLInOutBase.CPP  ************************
// *****************************************************************************
#include "CLInOutBase.h"

void CLInOutBase :: vOutPort(USI uiAdresse, UC ucData)
{
UC xdata *ucpPointeur;         //Initialisation pointeur sur memoire externe

ucpPointeur  = (UC xdata *) uiAdresse;// Adressage de la case memoire
*ucpPointeur = ucData;                // Ecriture de bits sur le bus de donnees
}

UC CLInOutBase :: ucInPort(USI uiAdresse)
{
UC xdata *ucpPointeur;         //Initialisation pointeur sur memoire externe
UC ucData   = 0;               //Initialisation d'une variable de reception
ucpPointeur = (UC xdata *)uiAdresse; //Adressage de la case memoire
ucData      = *ucpPointeur;          //Lecture de bits sur le bus de donnees
return ucData;
}

void CLInOutBase :: vOut(USI uiAdresse, UC ucData)
{
vOutPort(uiAdresse, ucData);
}

void CLInOutBase :: vOut(USI uiAdresse, USI uiDonnee, UC ucTypeTransfert)
{
union UNEntierOctet unData;
unData.uiEntier = uiDonnee;

if(ucTypeTransfert == SAMELSB)
  {
   vOutPort(uiAdresse, unData.stDoubleOctet.ucOctet1);
   vOutPort(uiAdresse, unData.stDoubleOctet.ucOctet1);
  }

if(ucTypeTransfert == SAMEMSB)
  {
   vOutPort(uiAdresse, unData.stDoubleOctet.ucOctet2);
   vOutPort(uiAdresse, unData.stDoubleOctet.ucOctet2);
  }

if(ucTypeTransfert == SUITELSB)
  {
   vOutPort(uiAdresse, unData.stDoubleOctet.ucOctet1);
   vOutPort(uiAdresse, unData.stDoubleOctet.ucOctet2);
  }

if(ucTypeTransfert == SUITEMSB)
  {
   vOutPort(uiAdresse, unData.stDoubleOctet.ucOctet2);
   vOutPort(uiAdresse, unData.stDoubleOctet.ucOctet1);
  }
}

UC CLInOutBase :: ucIn(USI uiAdresse)
{
UC ucData = 0;                 //Initialisation d'une variable de reception
ucData    = ucInPort(uiAdresse);
return ucData;
}

UC CLInOutBase :: ucIn(USI uiAdresse, UC ucTypeTransfert)
{
union UNEntierOctet unData;
unData.uiEntier = 0x00;

unData.stDoubleOctet.ucOctet1 = 0x00;
unData.stDoubleOctet.ucOctet2 = 0x00;

if(ucTypeTransfert == SAMELSB)
  {
   unData.stDoubleOctet.ucOctet1 = ucInPort(uiAdresse);
   unData.stDoubleOctet.ucOctet1 = ucInPort(uiAdresse);
  }

if(ucTypeTransfert == SAMEMSB)
  {
   unData.stDoubleOctet.ucOctet2 = ucInPort(uiAdresse);
   unData.stDoubleOctet.ucOctet2 = ucInPort(uiAdresse);
  }

if(ucTypeTransfert == SUITELSB)
   {
    unData.stDoubleOctet.ucOctet1 = ucInPort(uiAdresse);
    unData.stDoubleOctet.ucOctet2 = ucInPort(uiAdresse);
   }

if(ucTypeTransfert == SUITEMSB)
  {
   unData.stDoubleOctet.ucOctet2 = ucInPort(uiAdresse);
   unData.stDoubleOctet.ucOctet1 = ucInPort(uiAdresse);
  }
return unData.uiEntier;
}
//LinuxLinuxLinuxLinuxLinuxLinuxLinuxLinuxLinuxLinuxLinuxLinuxLinuxLinuxLinuxTUX

// **************************Main.c
// Auteur:      Vincent Chouinard
// Date:        5 aout Janvier 2014
// Version :    1.0
//
// Compilateur: Keil 4.72.9 && IAR 8.10
//
// Description: Fichier principal
//
// *****************************************************************************
// **** LES INCLUDES ****************//
#include "_DeclarationGenerale.h"    // Raccourcis Linguistiques utiles
#include "ConversionKeilToIAR.h"     //
#include "CLClavier.h"               // Pour utiliser le clavier
#include "CLEcran.h"                 // Pour utiliser l'écran
#include "NomDuProjet.h"             //
#include "CLCommunic.h"              // Pour utiliser le port série
// **** LES FONCTIONS DU MAIN *******//
void vRecuFromKeyBoard(void);
void vRecuFromRS232   (void);
// **** CLASS DECLARATION ***********//
class CLEcran    clEcran(0xF800);
class CLClavier  Clavier;
class CLCommunic clCommunic;
// **** PROGRAMME PRINCIPAL *********//
void main(void)
{
EA    = 0;     // Disable interrupt.
WDKEY = 0x55;  // Disable WatchDog.
EA    = 1;

while(1)
  {
   if(Clavier.ucBoutonAppuye() == ACTIF)
     {
      vRecuFromKeyBoard();
      clEcran.vLCDCursor(0,1);
      clEcran.vLCDDisplayCarac(Clavier.ucLireClavier());
      clCommunic.vTransCaratere(Clavier.ucLireClavier());
     }

   if(clCommunic.ucLireEtatReception() == ACTIF)
     {
      vRecuFromRS232();
      clEcran.vLCDCursor(0,2);
      clEcran.vLCDDisplayCarac(clCommunic.ucLireTrameSansCheck());
     }
  }
}


void vRecuFromKeyBoard(void)
{
clEcran.vLCDCursor(3,1);
clEcran.vLCDDisplayCarac('R');
clEcran.vLCDDisplayCarac('e');
clEcran.vLCDDisplayCarac('c');
clEcran.vLCDDisplayCarac('u');
clEcran.vLCDDisplayCarac(' ');
clEcran.vLCDDisplayCarac('K');
clEcran.vLCDDisplayCarac('e');
clEcran.vLCDDisplayCarac('y');
clEcran.vLCDDisplayCarac('b');
clEcran.vLCDDisplayCarac('o');
clEcran.vLCDDisplayCarac('a');
clEcran.vLCDDisplayCarac('r');
clEcran.vLCDDisplayCarac('d');
}

void vRecuFromRS232(void)
{
clEcran.vLCDCursor(3,2);
clEcran.vLCDDisplayCarac('R');
clEcran.vLCDDisplayCarac('e');
clEcran.vLCDDisplayCarac('c');
clEcran.vLCDDisplayCarac('u');
clEcran.vLCDDisplayCarac(' ');
clEcran.vLCDDisplayCarac('R');
clEcran.vLCDDisplayCarac('S');
clEcran.vLCDDisplayCarac('2');
clEcran.vLCDDisplayCarac('3');
clEcran.vLCDDisplayCarac('2');
}

//LinuxLinuxLinuxLinuxLinuxLinuxLinuxLinuxLinuxLinuxLinuxLinuxLinuxLinuxLinuxTUX
// **************************Clavier.h
// Auteur:       Vincent Chouinard
// Date:         1 mai 2014
// Version:      1.0
// Modification: Aucune
//
// Compilateur:  IAR 8.1
//
// Description:  Fichier de definitions pour utiliser le clavier du uPSD
// *****************************************************************************
///////////////////////////////////////////////////////////////////////////////
//                     Infos relatives au montage du clavier sur le uPSD
// Utilise un 74HCC922
// 74HCC922      uPSD
//        A sur D0
//        B sur D1
//        C sur D2
//        D sur D3
//       OE sur CS1     (OutputEnable sur chip select 1)
//       DA sur INT0    (AataAvailible sur interrupt0)
///////////////////////////////////////////////////////////////////////////////

#include "_DeclarationGenerale.h" // Raccourcis de programmation & variables
#include "CLInOutBase.h"          // Fichier de definitions pour faire des I/O

#ifndef CLAVIERH
  #define CLAVIERH

  #define CS1 0xF900     // Chip Select 1 du uPSD sur 0xF900

class CLClavier : public CLInOutBase
{
public:
   CLClavier(void);                    // Constructeur par defaut.
  ~CLClavier(void);                    // Desctructeur.

   UC ucLireClavier(void);
   UC ucBoutonAppuye(void);            // Fonction de logique de selection du bouton pese

protected:
   UC        ucToucheClavier;
   static UC ucEtatClavier;

private:
   __interrupt static void vKeyboardInterrupt(void);
};
#endif
//LinuxLinuxLinuxLinuxLinuxLinuxLinuxLinuxLinuxLinuxLinuxLinuxLinuxLinuxLinuxTUX
//**********************  Fichier: CLCommunic.h
//  Description : Fichier d'entete pour le fichier CLCommunic.cpp
//
//  Composition: aucune
//  Heritage : aucun
//  Fonctions necessaire pour communiquer avec le port serie:
//
//CLCommunic()                                          Constructeur par defaut
//CLCommunic(UC ucLongBuf, UC ucLongTrame, UI uiBaud)  Construct initialisateur
//~CLCommunic()                                                     Destructeur
//UC ucInitLongTrame(UC ucLong)                 Pour initialiser longueur trame
//UC static ucLireNbCaractRecu()         Fct pour lire nombre de caractere recu
//UC ucLireEtatTrame()                         Fct pour lire l'etat de la trame
//char *ucpLireTrame()                Fct qui retourne adresse du debut lecture
//vTransCaratere(UC ucCar)                    Fct pour transmettre un caractere
//vTransChiffre(UI uiChiffreAAfficher)          Fct pour transmettre un chiffre
//vAffChaine(char const *ucpMessage)   Fct pour transmettre chaine de caractere
//void vInitBaudRate(UI uiBaud)        Fct pour intialiser le baudrate et timer
//__interrupt static void vInteruptSerial();     Fct pour recevoir un caractere
//UC ucValideTrame()                                 Fct pour valider une trame

//
//  Programmeur: Philippe Dubois
//  Cours: 247-436
//
//  Date: 10 avril 2014
//                                                                           //
//  Compilateur: IAR 8.1
//                                                                           //
//  Modification:                                                            //
//                                                                           //
///////////////////////////////////////////////////////////////////////////////
#ifndef CLCOMMUNICH               // Si CLCOMMUNICH non defini.
   #define CLCOMMUNICH            // Defini CLCOMMUNICH

#include <stdio.h>               // Definition des function I/O (printf, getchar
#include "_DeclarationGenerale.h" // Pour les declaration generale
#include "TypeCible.h"            // Pour les definition des registre.
#include "ConversionKeilToIAR.h"  //

#define LONGUEURTRAME  4          // Longueur trame par defaut.
#define LONGUEURTAMPON 12         // Longueur tampon par defaut.
#define TRAMECOMPLET   1          // Pour l'etat de la trame.
#define TRAMEINCOMPLET 0          // Pour l'etat de la trame.

class CLCommunic            // Classe CLCommunic
{
public:                     // Menbres publics.
   CLCommunic(void);                                    // Constructeur par defaut.
   CLCommunic(UC ucLongBuf, UC ucLongTrame, UI uiBaud); // Constructeur initialisateur
  ~CLCommunic(void);                                   // Destructeur

UC    ucLireEtatReception(void);
UC    ucLireTrameSansCheck(void);
UC    ucInitLongTrame(UC ucLong);           // Intialiser la longueur trame.
UC    static ucLireNbCaractRecu(void);      // Lire le nombre de caractere recu
UC    ucLireEtatTrame(void);                // Lire l'etat de la trame.
char *ucpLireTrame(void);                   // Retourne ladresse du tableau.
void  vTransCaratere(UC ucCar);             // Transmettre un caractere.
void  vTransChiffre(UI uiChiffreAAfficher); // Transmettre un chiffre sur le port serie
void  vAffChaine(char const *ucpMessage);   // Transmettre une chaine de caractereS

protected:

private:                      // Menbres prives.
void vInitBaudRate(UI uiBaud);// Fonction pour intialiser le baudrate et timer.

__interrupt static void vInteruptSerial(void);// Fonction d'interruption pour
                                              // recevoir un caractere.
UC ucValideTrame(void);                       // Fct pour valide une trame.

static char *cTabCaractereRecu;     // Pour pointeur sur tableau de char.
static char *cpDebutTrame;          // Pointeur sur debut tampon.
static char *cpFinTrame;            // Pointeur sur fin tampon.
static char *cptrEcriture;          // Pointeur pour position ecriture.
static char *cptrLecture;           // Pointeur pour lecture.
static UC    ucEtatTrame;           // Pour connaitre l'etat de la trame
static UC    ucNbCaractRecu;        // Pour connaitre le nbr de caractere recu.
static UC    ucLongueurTrame;       // Pour connaitre la longueur de la trame.
static UC    ucLongueurTampon;      // Pour connaitre la longueur du tampon.
static UC    ucCaractereRecu;
static UC    ucEtatReception;
};
#endif
//LinuxLinuxLinuxLinuxLinuxLinuxLinuxLinuxLinuxLinuxLinuxLinuxLinuxLinuxLinuxTUX
















// ************************** FICHIER: CLEcran.H  ******************************

// DESCRIPTION DES FONCTIONS
//
// CLEcran(void);
// CLEcran(USI uiAdresse);
// ~CLEcran(void);
//
// void vLCDClr(void);
// void vLCDInit(void);
// void vLCDCursor(UC ucCol, UC ucLine);
//
// void vLCDDisplayCarac(UC ucCaractere);
// void vLCDDisplayCarac(char cCaractere);
//
// void vLCDDisplayCaracChain(const UC *ucpMessage);
// void vLCDDisplayCaracChain(const char *cpMessage);
//
// void vHexToASCII(UC ucHexCar, UC *ucpHi, UC *ucpLo);
// void vLCDDisplayHexCarac(UC ucHexCar);
//
// void vLCDDisplayScreen(const UC *ucpEcran);
// void vLCDBusy(void);
// void vDelaiLCD(USI uiDelai);
//
// *****************************************************************************
#include "_DeclarationGenerale.h"
#include "CLInOutBase.h"

#ifndef CLECRANH
   #define CLECRANH

#define ECRANCONFIG 0
#define ECRANFLAG   1
#define ECRANWR     2
#define ECRANRD     3

class CLEcran : public CLInOutBase
{
public:
   CLEcran(void);
   CLEcran(USI uiAdresse);
  ~CLEcran(void);

   void vLCDClr              (void);
   void vLCDInit             (void);
   void vLCDCursor           (UC   ucCol, UC ucLine);
   void vLCDDisplayCarac     (UC   ucCaractere);
   void vLCDDisplayCarac     (char cCaractere);

   void vHexToASCII          (UC   ucHexCar,   UC *ucpHi,      UC *ucpLo);
   void vHexToASCII          (USI  uiHexCar,   UC *ucpHiOctHi, UC *ucpLoOctHi,
                              UC  *ucpHiOctLo, UC *ucpLoOctLo);

   void vLCDDisplayHexCarac  (UC          ucHexCar);
   void vLCDDisplayHexCarac  (USI         uiHexCar);
   void vLCDDisplayCaracChain(const UC   *ucpMessage);
   void vLCDDisplayCaracChain(const char *cpMessage);
   void vLCDDisplayScreen    (const UC   *ucpEcran);
   void vLCDDisplayEtatPort  (UC          ucEtatPort);
   void vLCDDisplayFloat     (float       fInputFloat);

protected:

private:
   USI  uiAdresseEcran;
   void vLCDBusy (void);
   void vDelaiLCD(USI uiDelai);
};
#endif
//LinuxLinuxLinuxLinuxLinuxLinuxLinuxLinuxLinuxLinuxLinuxLinuxLinuxLinuxLinuxTUX
// ************************** FICHIER: CLInOutBase.H  **************************
// Auteur:       Vincent Chouinard
// Date:         1 mai 2014
// Version:      1.0
// Modification: Aucune
//
// Compilateur:  IAR 8.1
//
// Description:  Fichier de definitions pour utiliser le clavier du uPSD
// *****************************************************************************
#include "_DeclarationGenerale.h"

#ifndef CLINOUTBASEH
   #define CLINOUTBASEH

#define SAMELSB  1
#define SAMEMSB  2
#define SUITELSB 3
#define SUITEMSB 4

class CLInOutBase
{
public:

protected:
   void vOut(USI uiAdresse, UC ucData);
   void vOut(USI uiAdresse, USI uiDonnee, UC ucTypeTransfert);

   UC   ucIn(USI uiAdresse);
   UC   ucIn(USI uiAdresse, UC ucTypeTransfert);

private:
   UC   ucInPort(USI uiAdresse);
   void vOutPort(USI uiAdresse, UC ucData);
};
#endif
//LinuxLinuxLinuxLinuxLinuxLinuxLinuxLinuxLinuxLinuxLinuxLinuxLinuxLinuxLinuxTUX

// ************************** FICHIER: ConversionKeilToIAR.h *******************
//
// Auteur:       Vincent Chouinard
// Date:         1 mai 2014
// Version:      1.0
// Modification: Aucune
//
// Compilateur:  IAR 8.1 et Keil 4.0
//
// Description: Conversion des SFR entre Keil 4.0 et IAR 8.1
// *****************************************************************************

#include "TypeCible.h"

#ifndef CONVERSIONKEILTOIARH
   #define CONVERSIONKEILTOIARH

// *****************************************************************************
// DEFINITION DES  MODELES DE MEMOIRE
// *****************************************************************************

#define data  __data
#define xdata __xdata
#define code  __code


// *****************************************************************************
#ifdef DALLAS89C450

// Interruption activation
  #define EA            IE_bit.EA
  #define ES0           IE_bit.ES0
  #define ES2           IE_bit.ES1
  #define ET1           IE_bit.ET1
  #define EX0           IE_bit.EX0

// Interruption priorite
  #define LPX0          IP0_bit.LPX0
  #define MPX0          IP1_bit.MPX0

  #define LPS           IP0_bit.LPS0
  #define MPS           IP1_bit.MPS0
  #define LPS2          IP0_bit.LPS1
  #define MPS2          IP1_bit.MPS1

// Interruption niveau activite.
  #define IT0           TCON_bit.IT0
  #define TR1           TCON_bit.TR1

  #define TCLK2         T2CON_bit.TCKL
  #define TRCLK2        T2CON_bit.RCLK
  #define TR2           T2CON_bit.TR2

  #define T2MOD_DCEN    T2MOD_bit.DCEN

  #define SMOD          PCON_bit.SMOD_0
  #define SMOD2         WDCON_bit.SMOD_1

  #define SCON          SCON0
  #define SBUF          SBUF0
  #define RI_0          SCON0_bit.RI_0
  #define TI_0          SCON0_bit.TI_0

  #define SCON2         SCON1
  #define SBUF2         SBUF1
  #define RI_2          SCON1_bit.RI_1
  #define TI_2          SCON1_bit.TI_1

  #define SM0_0         SCON0_bit.SM0_FE_0
  #define SM1_0         SCON0_bit.SM1_0
  #define SM2_0         SCON0_bit.SM2_0
  #define REN_0         SCON0_bit.REN_0

  #define SM0_2         SCON1_bit.SM0_FE_1
  #define SM1_2         SCON1_bit.SM1_1
  #define SM2_2         SCON1_bit.SM2_1
  #define REN_2         SCON1_bit.REN_1

  #define P1_0          P1_bit.P10_T2          // Port 1.0
  #define P1_1          P1_bit.P11_T2EX        // Port 1.1
  #define P1_2          P1_bit.P12_RXD1        // Port 1.2
  #define P1_3          P1_bit.P13_TXD1        // Port 1.3
  #define P1_4          P1_bit.P14_INT2        // Port 1.4
  #define P1_5          P1_bit.P15_INT3        // Port 1.5
  #define P1_6          P1_bit.P16_INT4        // Port 1.6
  #define P1_7          P1_bit.P17_INT5        // Port 1.7

  #define P3_0          P3_bit.P30_RXD0        // Port 3.0
  #define P3_1          P3_bit.P31_TXD0        // Port 3.1
  #define P3_2          P3_bit.P32_INT0        // Port 3.2
  #define P3_3          P3_bit.P33_INT1        // Port 3.3
  #define P3_4          P3_bit.P34_T0          // Port 3.4
  #define P3_5          P3_bit.P35_T1          // Port 3.5
  #define P3_6          P3_bit.P36_WR          // Port 3.6
  #define P3_7          P3_bit.P37_RD          // Port 3.7

  #define SDA           P3_4
  #define SCL           P3_5
#endif // DALLAS89C450
// *****************************************************************************

// *****************************************************************************
#ifdef UPSD3254A
// Interruption activation
  #define EA            IE_bit.EA
  #define ES            IE_bit.ES
  #define ET1           IE_bit.ET1
  #define ES2           IEA_bit.ES2
  #define EX0           IE_bit.EX0

// Interruption priorite
  #define PX0           IP_bit.PX0
  #define PS            IP_bit.PS
  #define PS2           IPA_bit.PS2

// Interruption niveau activite.
  #define IT0           TCON_bit.IT0
  #define TR1           TCON_bit.TR1

  #define TCLK2         T2CON_bit.TCKL
  #define TRCLK2        T2CON_bit.RCLK
  #define TR2           T2CON_bit.TR2

  #define T2MOD_DCEN    T2MOD_bit.DCEN

  #define SBUF0         SBUF
  #define SCON0         SCON

 // #define ES0 IE_bit.ES
 // #define ES0 IE_bit.ES

  #define SMOD          PCON_bit.SMOD
  #define SMOD2         PCON_bit.SMOD1

  #define RI_0          SCON_bit.RI
  #define TI_0          SCON_bit.TI
  #define SM0_0         SCON_bit.SM0
  #define SM1_0         SCON_bit.SM1
  #define SM2_0         SCON_bit.SM2
  #define REN_0         SCON_bit.REN

  #define TI_2          SCON2_bit.TI
  #define RI_2          SCON2_bit.RI
  #define SM0_2         SCON2_bit.SM0
  #define SM1_2         SCON2_bit.SM1
  #define SM2_2         SCON2_bit.SM2
  #define REN_2         SCON2_bit.REN

  #define WATCHDOG_OFF WDKEY = 0x55; // Disable watchdog.

  #define P1_0          P1_bit.P10        // Port 1.0
  #define P1_1          P1_bit.P11        // Port 1.1
  #define P1_2          P1_bit.P12        // Port 1.2
  #define P1_3          P1_bit.P13        // Port 1.3
  #define P1_4          P1_bit.P14        // Port 1.4
  #define P1_5          P1_bit.P15        // Port 1.5
  #define P1_6          P1_bit.P16        // Port 1.6
  #define P1_7          P1_bit.P17        // Port 1.7

  #define P3_0          P3_bit.P30        // Port 3.0
  #define P3_1          P3_bit.P31        // Port 3.1
  #define P3_2          P3_bit.P32        // Port 3.2
  #define P3_3          P3_bit.P33        // Port 3.3
  #define P3_4          P3_bit.P34        // Port 3.4
  #define P3_5          P3_bit.P35        // Port 3.5
  #define P3_6          P3_bit.P36        // Port 3.6
  #define P3_7          P3_bit.P37        // Port 3.7

  #define P4_0          P4_bit.P40        // Port 4.0
  #define P4_1          P4_bit.P41        // Port 4.1
  #define P4_2          P4_bit.P42        // Port 4.2
  #define P4_3          P4_bit.P43        // Port 4.3
  #define P4_4          P4_bit.P44        // Port 4.4
  #define P4_5          P4_bit.P45        // Port 4.5
  #define P4_6          P4_bit.P46        // Port 4.6
  #define P4_7          P4_bit.P47        // Port 4.7

  #define SCL           P3_7
  #define SDA           P3_6





#endif // UPSD3254A
// *****************************************************************************
#endif // CONVERSIONKEILTOIARH
//LinuxLinuxLinuxLinuxLinuxLinuxLinuxLinuxLinuxLinuxLinuxLinuxLinuxLinuxLinuxTUX

/***************************************************************************
 *                                - ioDS89C450.h -
 *
 * Special header for the Maxim (Dallas Semiconductor) DS89C450 Microcontroller.
 *
 ***************************************************************************/

#ifndef IODS89C450_H
#define IODS89C450_H
#define __DS89C450__
#ifdef __IAR_SYSTEMS_ICC__
#ifndef _SYSTEM_BUILD
   #pragma system_include
#endif
#pragma language=extended

/*-------------------------------------------------------------------------
 *   All SFRs
 *-------------------------------------------------------------------------*/

__sfr __no_init volatile union
{
  unsigned char EIP1;
  struct
  {
    unsigned char MPX2 : 1;
    unsigned char MPX3 : 1;
    unsigned char MPX4 : 1;
    unsigned char MPX5 : 1;
    unsigned char MPWDI : 1;
    unsigned char  : 1;
    unsigned char  : 1;
    unsigned char  : 1;
  } EIP1_bit;
} @ 0xF1;
__sfr __no_init volatile unsigned char ACC @ 0xE0;
__sfr __no_init volatile unsigned char TL2 @ 0xCC;
__sfr __no_init volatile union
{
  unsigned char T2MOD;
  struct
  {
    unsigned char DCEN : 1;
    unsigned char T2OE : 1;
    unsigned char  : 1;
    unsigned char  : 1;
    unsigned char  : 1;
    unsigned char  : 1;
    unsigned char  : 1;
    unsigned char  : 1;
  } T2MOD_bit;
} @ 0xC9;
__sfr __no_init volatile union
{
  unsigned char IP0;
  struct
  {
    unsigned char LPX0 : 1;
    unsigned char LPT0 : 1;
    unsigned char LPX1 : 1;
    unsigned char LPT1 : 1;
    unsigned char LPS0 : 1;
    unsigned char LPT2 : 1;
    unsigned char LPS1 : 1;
    unsigned char  : 1;
  } IP0_bit;
} @ 0xB8;
__sfr __no_init volatile unsigned char SADDR1 @ 0xAA;
__sfr __no_init volatile union
{
  unsigned char ACON;
  struct
  {
    unsigned char  : 1;
    unsigned char  : 1;
    unsigned char  : 1;
    unsigned char  : 1;
    unsigned char  : 1;
    unsigned char PAGES0 : 1;
    unsigned char PAGES1 : 1;
    unsigned char PAGEE : 1;
  } ACON_bit;
} @ 0x9D;
__sfr __no_init volatile unsigned char SBUF0 @ 0x99;
__sfr __no_init volatile unsigned char TH0 @ 0x8C;
__sfr __no_init volatile union
{
  unsigned char TCON;
  struct
  {
    unsigned char IT0 : 1;
    unsigned char IE0 : 1;
    unsigned char IT1 : 1;
    unsigned char IE1 : 1;
    unsigned char TR0 : 1;
    unsigned char TF0 : 1;
    unsigned char TR1 : 1;
    unsigned char TF1 : 1;
  } TCON_bit;
} @ 0x88;
__sfr __no_init volatile union
{
  unsigned char PSW;
  struct
  {
    unsigned char P : 1;
    unsigned char F1 : 1;
    unsigned char OV : 1;
    unsigned char RS0 : 1;
    unsigned char RS1 : 1;
    unsigned char F0 : 1;
    unsigned char AC : 1;
    unsigned char CY : 1;
  } PSW_bit;
} @ 0xD0;
__sfr __no_init volatile unsigned char TH2 @ 0xCD;
__sfr __no_init volatile unsigned char SADEN0 @ 0xB9;
__sfr __no_init volatile union
{
  unsigned char IE;
  struct
  {
    unsigned char EX0 : 1;
    unsigned char ET0 : 1;
    unsigned char EX1 : 1;
    unsigned char ET1 : 1;
    unsigned char ES0 : 1;
    unsigned char ET2 : 1;
    unsigned char ES1 : 1;
    unsigned char EA : 1;
  } IE_bit;
} @ 0xA8;
__sfr __no_init volatile union
{
  unsigned char P1;
  struct
  {
    unsigned char P10_T2 : 1;
    unsigned char P11_T2EX : 1;
    unsigned char P12_RXD1 : 1;
    unsigned char P13_TXD1 : 1;
    unsigned char P14_INT2 : 1;
    unsigned char P15_INT3 : 1;
    unsigned char P16_INT4 : 1;
    unsigned char P17_INT5 : 1;
  } P1_bit;
} @ 0x90;
__sfr __no_init volatile unsigned char TH1 @ 0x8D;
__sfr __no_init volatile union
{
  unsigned char TMOD;
  struct
  {
    unsigned char M00 : 1;
    unsigned char M10 : 1;
    unsigned char C_T0 : 1;
    unsigned char GATE0 : 1;
    unsigned char M01 : 1;
    unsigned char M11 : 1;
    unsigned char C_T1 : 1;
    unsigned char GATE1 : 1;
  } TMOD_bit;
} @ 0x89;
__sfr __no_init volatile union
{
  unsigned char SCON1;
  struct
  {
    unsigned char RI_1 : 1;
    unsigned char TI_1 : 1;
    unsigned char RB8_1 : 1;
    unsigned char TB8_1 : 1;
    unsigned char REN_1 : 1;
    unsigned char SM2_1 : 1;
    unsigned char SM1_1 : 1;
    unsigned char SM0_FE_1 : 1;
  } SCON1_bit;
} @ 0xC0;
__sfr __no_init volatile unsigned char SADDR0 @ 0xA9;
__sfr __no_init volatile union
{
  unsigned char EXIF;
  struct
  {
    unsigned char BGS : 1;
    unsigned char RGSL : 1;
    unsigned char RGMD : 1;
    unsigned char CKRY : 1;
    unsigned char IE2 : 1;
    unsigned char IE3 : 1;
    unsigned char IE4 : 1;
    unsigned char IE5 : 1;
  } EXIF_bit;
} @ 0x91;
__sfr __no_init volatile union
{
  unsigned char CKCON;
  struct
  {
    unsigned char MD0 : 1;
    unsigned char MD1 : 1;
    unsigned char MD2 : 1;
    unsigned char T0M : 1;
    unsigned char T1M : 1;
    unsigned char T2M : 1;
    unsigned char WD0 : 1;
    unsigned char WD1 : 1;
  } CKCON_bit;
} @ 0x8E;
__sfr __no_init volatile union
{
  unsigned char P0;
  struct
  {
    unsigned char P00 : 1;
    unsigned char P01 : 1;
    unsigned char P02 : 1;
    unsigned char P03 : 1;
    unsigned char P04 : 1;
    unsigned char P05 : 1;
    unsigned char P06 : 1;
    unsigned char P07 : 1;
  } P0_bit;
} @ 0x80;
__sfr __no_init volatile unsigned char SBUF1 @ 0xC1;
__sfr __no_init volatile union
{
  unsigned char P3;
  struct
  {
    unsigned char P30_RXD0 : 1;
    unsigned char P31_TXD0 : 1;
    unsigned char P32_INT0 : 1;
    unsigned char P33_INT1 : 1;
    unsigned char P34_T0 : 1;
    unsigned char P35_T1 : 1;
    unsigned char P36_WR : 1;
    unsigned char P37_RD : 1;
  } P3_bit;
} @ 0xB0;
__sfr __no_init volatile unsigned char SP @ 0x81;
__sfr __no_init volatile union
{
  unsigned char ROMSIZE;
  struct
  {
    unsigned char RMS0 : 1;
    unsigned char RMS1 : 1;
    unsigned char RMS2 : 1;
    unsigned char PRAME : 1;
    unsigned char  : 1;
    unsigned char  : 1;
    unsigned char  : 1;
    unsigned char  : 1;
  } ROMSIZE_bit;
} @ 0xC2;
__sfr __no_init volatile union
{
  unsigned char IP1;
  struct
  {
    unsigned char MPX0 : 1;
    unsigned char MPT0 : 1;
    unsigned char MPX1 : 1;
    unsigned char MPT1 : 1;
    unsigned char MPS0 : 1;
    unsigned char MPT2 : 1;
    unsigned char MPS1 : 1;
    unsigned char  : 1;
  } IP1_bit;
} @ 0xB1;
__sfr __no_init volatile union
{
  unsigned char P2;
  struct
  {
    unsigned char P20 : 1;
    unsigned char P21 : 1;
    unsigned char P22 : 1;
    unsigned char P23 : 1;
    unsigned char P24 : 1;
    unsigned char P25 : 1;
    unsigned char P26 : 1;
    unsigned char P27 : 1;
  } P2_bit;
} @ 0xA0;
__sfr __no_init volatile unsigned char DPL @ 0x82;
__sfr __no_init volatile unsigned char DPH @ 0x83;
__sfr __no_init volatile union
{
  unsigned char FCNTL;
  struct
  {
    unsigned char FC0 : 1;
    unsigned char FC1 : 1;
    unsigned char FC2 : 1;
    unsigned char FC3 : 1;
    unsigned char  : 1;
    unsigned char  : 1;
    unsigned char FERR : 1;
    unsigned char FBUSY : 1;
  } FCNTL_bit;
} @ 0xD5;
__sfr __no_init volatile union
{
  unsigned char PMR;
  struct
  {
    unsigned char DME0 : 1;
    unsigned char DME1 : 1;
    unsigned char ALEON : 1;
    unsigned char _4X__2X : 1;
    unsigned char CTM : 1;
    unsigned char SWB : 1;
    unsigned char CD0 : 1;
    unsigned char CD1 : 1;
  } PMR_bit;
} @ 0xC4;
__sfr __no_init volatile unsigned char DPL1 @ 0x84;
__sfr __no_init volatile union
{
  unsigned char EIP0;
  struct
  {
    unsigned char LPX2 : 1;
    unsigned char LPX3 : 1;
    unsigned char LPX4 : 1;
    unsigned char LPX5 : 1;
    unsigned char LPWDI : 1;
    unsigned char  : 1;
    unsigned char  : 1;
    unsigned char  : 1;
  } EIP0_bit;
} @ 0xF8;
__sfr __no_init volatile unsigned char FDATA @ 0xD6;
__sfr __no_init volatile union
{
  unsigned char STATUS;
  struct
  {
    unsigned char SPRA0 : 1;
    unsigned char SPTA0 : 1;
    unsigned char SPRA1 : 1;
    unsigned char SPTA1 : 1;
    unsigned char  : 1;
    unsigned char PIS0 : 1;
    unsigned char PIS1 : 1;
    unsigned char PIS2 : 1;
  } STATUS_bit;
} @ 0xC5;
__sfr __no_init volatile union
{
  unsigned char CKMOD;
  struct
  {
    unsigned char  : 1;
    unsigned char  : 1;
    unsigned char  : 1;
    unsigned char T0MH : 1;
    unsigned char T1MH : 1;
    unsigned char T2MH : 1;
    unsigned char  : 1;
    unsigned char  : 1;
  } CKMOD_bit;
} @ 0x96;
__sfr __no_init volatile unsigned char DPH1 @ 0x85;
__sfr __no_init volatile union
{
  unsigned char EIE;
  struct
  {
    unsigned char EX2 : 1;
    unsigned char EX3 : 1;
    unsigned char EX4 : 1;
    unsigned char EX5 : 1;
    unsigned char EWDI : 1;
    unsigned char  : 1;
    unsigned char  : 1;
    unsigned char  : 1;
  } EIE_bit;
} @ 0xE8;
__sfr __no_init volatile unsigned char RCAP2L @ 0xCA;
__sfr __no_init volatile union
{
  unsigned char DPS;
  struct
  {
    unsigned char SEL : 1;
    unsigned char  : 1;
    unsigned char  : 1;
    unsigned char  : 1;
    unsigned char AID : 1;
    unsigned char TSL : 1;
    unsigned char ID0 : 1;
    unsigned char ID1 : 1;
  } DPS_bit;
} @ 0x86;
__sfr __no_init volatile union
{
  unsigned char WDCON;
  struct
  {
    unsigned char RWT : 1;
    unsigned char EWT : 1;
    unsigned char WTRF : 1;
    unsigned char WDIF : 1;
    unsigned char PFI : 1;
    unsigned char EPFI : 1;
    unsigned char POR : 1;
    unsigned char SMOD_1 : 1;
  } WDCON_bit;
} @ 0xD8;
__sfr __no_init volatile unsigned char RCAP2H @ 0xCB;
__sfr __no_init volatile unsigned char TA @ 0xC7;
__sfr __no_init volatile unsigned char SADEN1 @ 0xBA;
__sfr __no_init volatile union
{
  unsigned char SCON0;
  struct
  {
    unsigned char RI_0 : 1;
    unsigned char TI_0 : 1;
    unsigned char RB8_0 : 1;
    unsigned char TB8_0 : 1;
    unsigned char REN_0 : 1;
    unsigned char SM2_0 : 1;
    unsigned char SM1_0 : 1;
    unsigned char SM0_FE_0 : 1;
  } SCON0_bit;
} @ 0x98;
__sfr __no_init volatile unsigned char TL0 @ 0x8A;
__sfr __no_init volatile union
{
  unsigned char PCON;
  struct
  {
    unsigned char IDLE : 1;
    unsigned char STOP : 1;
    unsigned char GF0 : 1;
    unsigned char GF1 : 1;
    unsigned char OFDE : 1;
    unsigned char OFDF : 1;
    unsigned char SMOD0 : 1;
    unsigned char SMOD_0 : 1;
  } PCON_bit;
} @ 0x87;
__sfr __no_init volatile unsigned char B @ 0xF0;
__sfr __no_init volatile union
{
  unsigned char T2CON;
  struct
  {
    unsigned char CP_RL2 : 1;
    unsigned char C_T2 : 1;
    unsigned char TR2 : 1;
    unsigned char EXEN2 : 1;
    unsigned char TCLK : 1;
    unsigned char RCLK : 1;
    unsigned char EXF2 : 1;
    unsigned char TF2 : 1;
  } T2CON_bit;
} @ 0xC8;
__sfr __no_init volatile unsigned char TL1 @ 0x8B;
/*
 * Interrupt Vectors
 */
#define IE0_int 0x03 /* External Interrupt 0 */
#define TF0_int 0x0B /* Timer 0 Overflow */
#define IE1_int 0x13 /* External Interrupt 1 */
#define TF1_int 0x1B /* Timer 1 Overflow */
#define RI_0_int 0x23 /* Serial Port 0 */
#define TI_0_int 0x23 /* Serial Port 0 */
#define TF2_int 0x2B /* Timer 2 Overflow */
#define EXF2_int 0x2B /* Timer 2 Overflow */
#define PFI_int 0x33 /* Power Fail */
#define RI_1_int 0x3B /* Serial Port 1 */
#define TI_1_int 0x3B /* Serial Port 1 */
#define IE2_int 0x43 /* External Interrupt 2 */
#define IE3_int 0x4B /* External Interrupt 3 */
#define IE4_int 0x53 /* External Interrupt 4 */
#define IE5_int 0x5B /* External Interrupt 5 */
#define WDIF_int 0x63 /* Watchdog */

#pragma language=default
#endif  /* __IAR_SYSTEMS_ICC__  */

/***************************************************************************
 *   Assembler definitions
 *
 *   The following SFRs are built in in the a8051.exe and can not be
 *   defined explicitly in this file:
 *     ACC,B,PSW,SP,DPL,DPH
 *   The PSW-bits are built-in in the a8051.exe
 *     OV,AC,F0,RS0,RS1,P
 *
 ***************************************************************************/

#ifdef __IAR_SYSTEMS_ASM__


/*-------------------------------------------------------------------------
 *   All SFRs
 *-------------------------------------------------------------------------*/

P0 DEFINE 0x80
P0_P00 DEFINE 0x80.0
P0_P01 DEFINE 0x80.1
P0_P02 DEFINE 0x80.2
P0_P03 DEFINE 0x80.3
P0_P04 DEFINE 0x80.4
P0_P05 DEFINE 0x80.5
P0_P06 DEFINE 0x80.6
P0_P07 DEFINE 0x80.7
DPL1 DEFINE 0x84
DPH1 DEFINE 0x85
DPS DEFINE 0x86
PCON DEFINE 0x87
TCON DEFINE 0x88
TCON_IT0 DEFINE 0x88.0
TCON_IE0 DEFINE 0x88.1
TCON_IT1 DEFINE 0x88.2
TCON_IE1 DEFINE 0x88.3
TCON_TR0 DEFINE 0x88.4
TCON_TF0 DEFINE 0x88.5
TCON_TR1 DEFINE 0x88.6
TCON_TF1 DEFINE 0x88.7
TMOD DEFINE 0x89
TL0 DEFINE 0x8A
TL1 DEFINE 0x8B
TH0 DEFINE 0x8C
TH1 DEFINE 0x8D
CKCON DEFINE 0x8E
P1 DEFINE 0x90
P1_P10_T2 DEFINE 0x90.0
P1_P11_T2EX DEFINE 0x90.1
P1_P12_RXD1 DEFINE 0x90.2
P1_P13_TXD1 DEFINE 0x90.3
P1_P14_INT2 DEFINE 0x90.4
P1_P15_INT3 DEFINE 0x90.5
P1_P16_INT4 DEFINE 0x90.6
P1_P17_INT5 DEFINE 0x90.7
EXIF DEFINE 0x91
CKMOD DEFINE 0x96
SCON0 DEFINE 0x98
SCON0_RI_0 DEFINE 0x98.0
SCON0_TI_0 DEFINE 0x98.1
SCON0_RB8_0 DEFINE 0x98.2
SCON0_TB8_0 DEFINE 0x98.3
SCON0_REN_0 DEFINE 0x98.4
SCON0_SM2_0 DEFINE 0x98.5
SCON0_SM1_0 DEFINE 0x98.6
SCON0_SM0_FE_0 DEFINE 0x98.7
SBUF0 DEFINE 0x99
ACON DEFINE 0x9D
P2 DEFINE 0xA0
P2_P20 DEFINE 0xA0.0
P2_P21 DEFINE 0xA0.1
P2_P22 DEFINE 0xA0.2
P2_P23 DEFINE 0xA0.3
P2_P24 DEFINE 0xA0.4
P2_P25 DEFINE 0xA0.5
P2_P26 DEFINE 0xA0.6
P2_P27 DEFINE 0xA0.7
IE DEFINE 0xA8
IE_EX0 DEFINE 0xA8.0
IE_ET0 DEFINE 0xA8.1
IE_EX1 DEFINE 0xA8.2
IE_ET1 DEFINE 0xA8.3
IE_ES0 DEFINE 0xA8.4
IE_ET2 DEFINE 0xA8.5
IE_ES1 DEFINE 0xA8.6
IE_EA DEFINE 0xA8.7
SADDR0 DEFINE 0xA9
SADDR1 DEFINE 0xAA
P3 DEFINE 0xB0
P3_P30_RXD0 DEFINE 0xB0.0
P3_P31_TXD0 DEFINE 0xB0.1
P3_P32_INT0 DEFINE 0xB0.2
P3_P33_INT1 DEFINE 0xB0.3
P3_P34_T0 DEFINE 0xB0.4
P3_P35_T1 DEFINE 0xB0.5
P3_P36_WR DEFINE 0xB0.6
P3_P37_RD DEFINE 0xB0.7
IP1 DEFINE 0xB1
IP0 DEFINE 0xB8
IP0_LPX0 DEFINE 0xB8.0
IP0_LPT0 DEFINE 0xB8.1
IP0_LPX1 DEFINE 0xB8.2
IP0_LPT1 DEFINE 0xB8.3
IP0_LPS0 DEFINE 0xB8.4
IP0_LPT2 DEFINE 0xB8.5
IP0_LPS1 DEFINE 0xB8.6
SADEN0 DEFINE 0xB9
SADEN1 DEFINE 0xBA
SCON1 DEFINE 0xC0
SCON1_RI_1 DEFINE 0xC0.0
SCON1_TI_1 DEFINE 0xC0.1
SCON1_RB8_1 DEFINE 0xC0.2
SCON1_TB8_1 DEFINE 0xC0.3
SCON1_REN_1 DEFINE 0xC0.4
SCON1_SM2_1 DEFINE 0xC0.5
SCON1_SM1_1 DEFINE 0xC0.6
SCON1_SM0_FE_1 DEFINE 0xC0.7
SBUF1 DEFINE 0xC1
ROMSIZE DEFINE 0xC2
PMR DEFINE 0xC4
STATUS DEFINE 0xC5
TA DEFINE 0xC7
T2CON DEFINE 0xC8
T2CON_CP_RL2 DEFINE 0xC8.0
T2CON_C_T2 DEFINE 0xC8.1
T2CON_TR2 DEFINE 0xC8.2
T2CON_EXEN2 DEFINE 0xC8.3
T2CON_TCLK DEFINE 0xC8.4
T2CON_RCLK DEFINE 0xC8.5
T2CON_EXF2 DEFINE 0xC8.6
T2CON_TF2 DEFINE 0xC8.7
T2MOD DEFINE 0xC9
RCAP2L DEFINE 0xCA
RCAP2H DEFINE 0xCB
TL2 DEFINE 0xCC
TH2 DEFINE 0xCD
FCNTL DEFINE 0xD5
FDATA DEFINE 0xD6
WDCON DEFINE 0xD8
WDCON_RWT DEFINE 0xD8.0
WDCON_EWT DEFINE 0xD8.1
WDCON_WTRF DEFINE 0xD8.2
WDCON_WDIF DEFINE 0xD8.3
WDCON_PFI DEFINE 0xD8.4
WDCON_EPFI DEFINE 0xD8.5
WDCON_POR DEFINE 0xD8.6
WDCON_SMOD_1 DEFINE 0xD8.7
EIE DEFINE 0xE8
EIE_EX2 DEFINE 0xE8.0
EIE_EX3 DEFINE 0xE8.1
EIE_EX4 DEFINE 0xE8.2
EIE_EX5 DEFINE 0xE8.3
EIE_EWDI DEFINE 0xE8.4
EIP1 DEFINE 0xF1
EIP0 DEFINE 0xF8
EIP0_LPX2 DEFINE 0xF8.0
EIP0_LPX3 DEFINE 0xF8.1
EIP0_LPX4 DEFINE 0xF8.2
EIP0_LPX5 DEFINE 0xF8.3
EIP0_LPWDI DEFINE 0xF8.4

#endif /* __IAR_SYSTEMS_ASM__*/
#endif /* IODS89C450_H */
/***************************************************************************
 *                                - iouPSD3254A.h -
 *
 * Special header for the STMicroelectronics uPSD3254A Microcontroller.
 *
 ***************************************************************************/

#ifndef IOUPSD3254A_H
#define IOUPSD3254A_H
#define __UPSD3254A__
#ifdef __IAR_SYSTEMS_ICC__
#ifndef _SYSTEM_BUILD
   #pragma system_include
#endif
#pragma language=extended

/*-------------------------------------------------------------------------
 *   8051 Core
 *-------------------------------------------------------------------------*/

__sfr __no_init volatile unsigned char ACC @ 0xE0; /* Accumulator */
__sfr __no_init volatile union
{
  unsigned char PSW; /* Program Status Word */
  struct /* Program Status Word */
  {
    unsigned char P : 1;
    unsigned char F1 : 1;
    unsigned char OV : 1;
    unsigned char RS0 : 1;
    unsigned char RS1 : 1;
    unsigned char FO : 1;
    unsigned char AC : 1;
    unsigned char CY : 1;
  } PSW_bit;
} @ 0xD0;
__sfr __no_init volatile unsigned char SP @ 0x81; /* Stack Ptr */
__sfr __no_init volatile unsigned char DPL @ 0x82; /* Data Ptr Low */
__sfr __no_init volatile unsigned char DPH @ 0x83; /* Data Ptr High */
__sfr __no_init volatile unsigned char B @ 0xF0; /* B Register */

/*-------------------------------------------------------------------------
 *   Interrupt
 *-------------------------------------------------------------------------*/

__sfr __no_init volatile union
{
  unsigned char IP; /* Interrupt Priority */
  struct /* Interrupt Priority */
  {
    unsigned char PX0 : 1;
    unsigned char PT0 : 1;
    unsigned char PX1 : 1;
    unsigned char PT1 : 1;
    unsigned char PS : 1;
    unsigned char PT2 : 1;
    unsigned char  : 1;
    unsigned char  : 1;
  } IP_bit;
} @ 0xB8;
__sfr __no_init volatile union
{
  unsigned char IEA; /* Interrupt Enable (2nd) */
  struct /* Interrupt Enable (2nd) */
  {
    unsigned char EUSB : 1;
    unsigned char EI2C : 1;
    unsigned char  : 1;
    unsigned char  : 1;
    unsigned char ES2 : 1;
    unsigned char  : 1;
    unsigned char  : 1;
    unsigned char EDDC : 1;
  } IEA_bit;
} @ 0xA7;
__sfr __no_init volatile union
{
  unsigned char IE; /* Interrupt Enable */
  struct /* Interrupt Enable */
  {
    unsigned char EX0 : 1;
    unsigned char ET0 : 1;
    unsigned char EX1 : 1;
    unsigned char ET1 : 1;
    unsigned char ES : 1;
    unsigned char ET2 : 1;
    unsigned char  : 1;
    unsigned char EA : 1;
  } IE_bit;
} @ 0xA8;
__sfr __no_init volatile union
{
  unsigned char IPA; /* Interrupt Priority (2nd) */
  struct /* Interrupt Priority (2nd) */
  {
    unsigned char PUSB : 1;
    unsigned char PI2C : 1;
    unsigned char  : 1;
    unsigned char  : 1;
    unsigned char PS2 : 1;
    unsigned char  : 1;
    unsigned char  : 1;
    unsigned char PDDC : 1;
  } IPA_bit;
} @ 0xB7;

/*-------------------------------------------------------------------------
 *   I2C Bus
 *-------------------------------------------------------------------------*/

__sfr __no_init volatile union
{
  unsigned char S2STA; /* I2C Bus Status */
  struct /* I2C Bus Status */
  {
    unsigned char SLV : 1;
    unsigned char ACK_R : 1;
    unsigned char Blost : 1;
    unsigned char Bbusy : 1;
    unsigned char  : 1;
    unsigned char Intr : 1;
    unsigned char Stop : 1;
    unsigned char GC : 1;
  } S2STA_bit;
} @ 0xDD;
__sfr __no_init volatile unsigned char S2ADR @ 0xDF; /* I2C address */
__sfr __no_init volatile unsigned char S1SETUP @ 0xD1; /* DDC I2C (S1) Setup */
__sfr __no_init volatile unsigned char S2SETUP @ 0xD2; /* I2C (S2) Setup */
__sfr __no_init volatile union
{
  unsigned char S2CON; /* I2C Bus Control Reg */
  struct /* I2C Bus Control Reg */
  {
    unsigned char CR0 : 1;
    unsigned char CR1 : 1;
    unsigned char AA : 1;
    unsigned char ADDR : 1;
    unsigned char STO : 1;
    unsigned char STA : 1;
    unsigned char EN1 : 1;
    unsigned char CR2 : 1;
  } S2CON_bit;
} @ 0xDC;
__sfr __no_init volatile union
{
  unsigned char S1CON; /* DDC I2C Control Reg */
  struct /* DDC I2C Control Reg */
  {
    unsigned char CR0 : 1;
    unsigned char CR1 : 1;
    unsigned char AA : 1;
    unsigned char ADDR : 1;
    unsigned char STO : 1;
    unsigned char STA : 1;
    unsigned char ENI1 : 1;
    unsigned char CR2 : 1;
  } S1CON_bit;
} @ 0xD8;
__sfr __no_init volatile union
{
  unsigned char S1STA; /* DDC I2C Status */
  struct /* DDC I2C Status */
  {
    unsigned char SLV : 1;
    unsigned char ACK_R : 1;
    unsigned char Blost : 1;
    unsigned char Bbusy : 1;
    unsigned char  : 1;
    unsigned char Intr : 1;
    unsigned char Stop : 1;
    unsigned char GC : 1;
  } S1STA_bit;
} @ 0xD9;

/*-------------------------------------------------------------------------
 *   USB
 *-------------------------------------------------------------------------*/

__sfr __no_init volatile union
{
  unsigned char UADR; /* USB Address Register */
  struct /* USB Address Register */
  {
    unsigned char UADD0 : 1;
    unsigned char UADD1 : 1;
    unsigned char UADD2 : 1;
    unsigned char UADD3 : 1;
    unsigned char UADD4 : 1;
    unsigned char UADD5 : 1;
    unsigned char UADD6 : 1;
    unsigned char USBEN : 1;
  } UADR_bit;
} @ 0xEE;
__sfr __no_init volatile union
{
  unsigned char UDR0; /* USB Endpt0 Data Recv */
  struct /* USB Endpt0 Data Recv */
  {
    unsigned char UDR00 : 1;
    unsigned char UDR01 : 1;
    unsigned char UDR02 : 1;
    unsigned char UDR03 : 1;
    unsigned char UDR04 : 1;
    unsigned char UDR05 : 1;
    unsigned char UDR06 : 1;
    unsigned char UDR07 : 1;
  } UDR0_bit;
} @ 0xEF;
__sfr __no_init volatile unsigned char USCL @ 0xE1; /* 8-bit Prescaler for USB logic */
__sfr __no_init volatile union
{
  unsigned char UCON0; /* USB Endpt0 Xmit Control */
  struct /* USB Endpt0 Xmit Control */
  {
    unsigned char TP0SIZ0 : 1;
    unsigned char TP0SIZ1 : 1;
    unsigned char TP0SiZ2 : 1;
    unsigned char TP0SIZ3 : 1;
    unsigned char RX0E : 1;
    unsigned char TX0E : 1;
    unsigned char STALL0 : 1;
    unsigned char TSEQ0 : 1;
  } UCON0_bit;
} @ 0xEA;
__sfr __no_init volatile union
{
  unsigned char UDT1; /* USB Endpt1 Data Xmit */
  struct /* USB Endpt1 Data Xmit */
  {
    unsigned char UDT10 : 1;
    unsigned char UDT11 : 1;
    unsigned char UDT12 : 1;
    unsigned char UDT13 : 1;
    unsigned char UDT14 : 1;
    unsigned char UDT15 : 1;
    unsigned char UDT16 : 1;
    unsigned char UDT17 : 1;
  } UDT1_bit;
} @ 0xE6;
__sfr __no_init volatile union
{
  unsigned char UCON1; /* USB Endpt1 Xmit Control */
  struct /* USB Endpt1 Xmit Control */
  {
    unsigned char TP1SIZ0 : 1;
    unsigned char TP1SIZ1 : 1;
    unsigned char TP1SiZ2 : 1;
    unsigned char TP1SIZ3 : 1;
    unsigned char FRESUM : 1;
    unsigned char  : 1;
    unsigned char EP12SEL : 1;
    unsigned char TSEQ1 : 1;
  } UCON1_bit;
} @ 0xEB;
__sfr __no_init volatile union
{
  unsigned char UDT0; /* USB Endpt0 Data Xmit */
  struct /* USB Endpt0 Data Xmit */
  {
    unsigned char UDT00 : 1;
    unsigned char UDT01 : 1;
    unsigned char UDT02 : 1;
    unsigned char UDT03 : 1;
    unsigned char UDT04 : 1;
    unsigned char UDT05 : 1;
    unsigned char UDT06 : 1;
    unsigned char UDT07 : 1;
  } UDT0_bit;
} @ 0xE7;
__sfr __no_init volatile union
{
  unsigned char UCON2; /* USB Control Register */
  struct /* USB Control Register */
  {
    unsigned char STALL1 : 1;
    unsigned char STALL2 : 1;
    unsigned char EP1E : 1;
    unsigned char EP2E : 1;
    unsigned char SOUT : 1;
    unsigned char  : 1;
    unsigned char  : 1;
    unsigned char  : 1;
  } UCON2_bit;
} @ 0xEC;
__sfr __no_init volatile union
{
  unsigned char UISTA; /* USB Interrupt Status */
  struct /* USB Interrupt Status */
  {
    unsigned char RESUMF : 1;
    unsigned char EOPF : 1;
    unsigned char RXD1F : 1;
    unsigned char RXD0F : 1;
    unsigned char TXD0F : 1;
    unsigned char RSTF : 1;
    unsigned char  : 1;
    unsigned char SUSPND : 1;
  } UISTA_bit;
} @ 0xE8;
__sfr __no_init volatile union
{
  unsigned char USTA; /* USB Endpt0 Status */
  struct /* USB Endpt0 Status */
  {
    unsigned char RP0SIZ0 : 1;
    unsigned char RP0SIZ1 : 1;
    unsigned char RP0SIZ2 : 1;
    unsigned char RP0SIZ3 : 1;
    unsigned char OUT : 1;
    unsigned char IN : 1;
    unsigned char SETUP : 1;
    unsigned char RSEQ : 1;
  } USTA_bit;
} @ 0xED;
__sfr __no_init volatile union
{
  unsigned char UIEN; /* USB Interrupt Enable */
  struct /* USB Interrupt Enable */
  {
    unsigned char RESUMIE : 1;
    unsigned char EOPIE : 1;
    unsigned char TXD1IE : 1;
    unsigned char RXD0IE : 1;
    unsigned char TXD0IE : 1;
    unsigned char RSTFIE : 1;
    unsigned char RSTE : 1;
    unsigned char SUSPNDIE : 1;
  } UIEN_bit;
} @ 0xE9;

/*-------------------------------------------------------------------------
 *   I/O Port
 *-------------------------------------------------------------------------*/

__sfr __no_init volatile union
{
  unsigned char P1; /* Port 1 */
  struct /* Port 1 */
  {
    unsigned char P10 : 1;
    unsigned char P11 : 1;
    unsigned char P12 : 1;
    unsigned char P13 : 1;
    unsigned char P14 : 1;
    unsigned char P15 : 1;
    unsigned char P16 : 1;
    unsigned char P17 : 1;
  } P1_bit;
} @ 0x90;
__sfr __no_init volatile union
{
  unsigned char P4; /* New Port 4 */
  struct /* New Port 4 */
  {
    unsigned char P40 : 1;
    unsigned char P41 : 1;
    unsigned char P42 : 1;
    unsigned char P43 : 1;
    unsigned char P44 : 1;
    unsigned char P45 : 1;
    unsigned char P46 : 1;
    unsigned char P47 : 1;
  } P4_bit;
} @ 0xC0;
__sfr __no_init volatile union
{
  unsigned char P1SFS; /* Port 1 Select Register */
  struct /* Port 1 Select Register */
  {
    unsigned char  : 1;
    unsigned char  : 1;
    unsigned char  : 1;
    unsigned char  : 1;
    unsigned char P1S4 : 1;
    unsigned char P1S5 : 1;
    unsigned char P1S6 : 1;
    unsigned char P1S7 : 1;
  } P1SFS_bit;
} @ 0x91;
__sfr __no_init volatile union
{
  unsigned char P0; /* Port 0 */
  struct /* Port 0 */
  {
    unsigned char P00 : 1;
    unsigned char P01 : 1;
    unsigned char P02 : 1;
    unsigned char P03 : 1;
    unsigned char P04 : 1;
    unsigned char P05 : 1;
    unsigned char P06 : 1;
    unsigned char P07 : 1;
  } P0_bit;
} @ 0x80;
__sfr __no_init volatile union
{
  unsigned char P3; /* Port 3 */
  struct /* Port 3 */
  {
    unsigned char P30 : 1;
    unsigned char P31 : 1;
    unsigned char P32 : 1;
    unsigned char P33 : 1;
    unsigned char P34 : 1;
    unsigned char P35 : 1;
    unsigned char P36 : 1;
    unsigned char P37 : 1;
  } P3_bit;
} @ 0xB0;
__sfr __no_init volatile union
{
  unsigned char P2; /* Port 2 */
  struct /* Port 2 */
  {
    unsigned char P20 : 1;
    unsigned char P21 : 1;
    unsigned char P22 : 1;
    unsigned char P23 : 1;
    unsigned char P24 : 1;
    unsigned char P25 : 1;
    unsigned char P26 : 1;
    unsigned char P27 : 1;
  } P2_bit;
} @ 0xA0;
__sfr __no_init volatile union
{
  unsigned char P3SFS; /* Port 3 Select Register */
  struct /* Port 3 Select Register */
  {
    unsigned char  : 1;
    unsigned char  : 1;
    unsigned char  : 1;
    unsigned char  : 1;
    unsigned char  : 1;
    unsigned char  : 1;
    unsigned char P3S6 : 1;
    unsigned char P3S7 : 1;
  } P3SFS_bit;
} @ 0x93;
__sfr __no_init volatile union
{
  unsigned char P4SFS; /* Port 4 Select Register */
  struct /* Port 4 Select Register */
  {
    unsigned char P4S0 : 1;
    unsigned char P4S1 : 1;
    unsigned char P4S2 : 1;
    unsigned char P4S3 : 1;
    unsigned char P4S4 : 1;
    unsigned char P4S5 : 1;
    unsigned char P4S6 : 1;
    unsigned char P4S7 : 1;
  } P4SFS_bit;
} @ 0x94;

/*-------------------------------------------------------------------------
 *   PWM
 *-------------------------------------------------------------------------*/

__sfr __no_init volatile unsigned char PWM4P @ 0xAA; /* PWM 4 Period */
__sfr __no_init volatile unsigned char PWM4W @ 0xAB; /* PWM 4 Pulse Width */
__sfr __no_init volatile union
{
  unsigned char PWMCON; /* PWM Control Polarity */
  struct /* PWM Control Polarity */
  {
    unsigned char CFG0 : 1;
    unsigned char CFG1 : 1;
    unsigned char CFG2 : 1;
    unsigned char CFG3 : 1;
    unsigned char CFG4 : 1;
    unsigned char PWME : 1;
    unsigned char PWMP : 1;
    unsigned char PWML : 1;
  } PWMCON_bit;
} @ 0xA1;
__sfr __no_init volatile unsigned char PWM1 @ 0xA3; /* PWM1 Output Duty Cycle */
__sfr __no_init volatile unsigned char PWM2 @ 0xA4; /* PWM2 Output Duty Cycle */
__sfr __no_init volatile unsigned char PWM3 @ 0xA5; /* PWM3 Output Duty Cycle */

/*-------------------------------------------------------------------------
 *   Serial I/O
 *-------------------------------------------------------------------------*/

__sfr __no_init volatile unsigned char SBUF @ 0x99; /* Serial Buffer */
__sfr __no_init volatile union
{
  unsigned char SCON2; /* 2nd UART Ctrl Register */
  struct /* 2nd UART Ctrl Register */
  {
    unsigned char RI : 1;
    unsigned char TI : 1;
    unsigned char RB8 : 1;
    unsigned char TB8 : 1;
    unsigned char REN : 1;
    unsigned char SM2 : 1;
    unsigned char SM1 : 1;
    unsigned char SM0 : 1;
  } SCON2_bit;
} @ 0x9A;
__sfr __no_init volatile unsigned char SBUF2 @ 0x9B; /* 2nd UART Serial Buffer */
__sfr __no_init volatile union
{
  unsigned char SCON; /* Serial Control Register */
  struct /* Serial Control Register */
  {
    unsigned char RI : 1;
    unsigned char TI : 1;
    unsigned char RB8 : 1;
    unsigned char TB8 : 1;
    unsigned char REN : 1;
    unsigned char SM2 : 1;
    unsigned char SM1 : 1;
    unsigned char SM0 : 1;
  } SCON_bit;
} @ 0x98;

/*-------------------------------------------------------------------------
 *   Watchdog timer
 *-------------------------------------------------------------------------*/

__sfr __no_init volatile unsigned char WDKEY @ 0xAE; /* Watch Dog Key Register */
__sfr __no_init volatile unsigned char WDRST @ 0xA6; /* Watch Dog Reset */

/*-------------------------------------------------------------------------
 *   System Management
 *-------------------------------------------------------------------------*/

__sfr __no_init volatile unsigned char S2DAT @ 0xDE; /* Data Hold Register */
__sfr __no_init volatile unsigned char PSCL0L @ 0xB1; /* Prescaler 0 Low (8-bit) */
__sfr __no_init volatile unsigned char RAMBUF @ 0xD4; /* DDC Ram Buffer */
__sfr __no_init volatile unsigned char PSCL0H @ 0xB2; /* Prescaler 0 High (8-bit) */
__sfr __no_init volatile unsigned char DDCDAT @ 0xD5; /* DDC Data xmit register */
__sfr __no_init volatile unsigned char PSCL1L @ 0xB3; /* Prescaler 1 Low (8-bit) */
__sfr __no_init volatile unsigned char PWM0 @ 0xA2; /* Output Duty Cycle */
__sfr __no_init volatile unsigned char S1DAT @ 0xDA; /* Data Hold Register */
__sfr __no_init volatile unsigned char DDCADR @ 0xD6; /* Addr pointer register */
__sfr __no_init volatile unsigned char PSCL1H @ 0xB4; /* Prescaler 1 High (8-bit) */
__sfr __no_init volatile unsigned char S1ADR @ 0xDB; /* DDC I2C address */
__sfr __no_init volatile union
{
  unsigned char DDCCON; /* DDC Control Register */
  struct /* DDC Control Register */
  {
    unsigned char M0 : 1;
    unsigned char SWHINT : 1;
    unsigned char DDC1EN : 1;
    unsigned char DDCINT : 1;
    unsigned char DDC_AX : 1;
    unsigned char SWENB : 1;
    unsigned char EX_DAT : 1;
    unsigned char  : 1;
  } DDCCON_bit;
} @ 0xD7;
__sfr __no_init volatile union
{
  unsigned char PCON; /* Power Ctrl */
  struct /* Power Ctrl */
  {
    unsigned char IDLE : 1;
    unsigned char PD : 1;
    unsigned char TCLK1 : 1;
    unsigned char RCLK1 : 1;
    unsigned char ADSFINT : 1;
    unsigned char LVREN : 1;
    unsigned char SMOD1 : 1;
    unsigned char SMOD : 1;
  } PCON_bit;
} @ 0x87;

/*-------------------------------------------------------------------------
 *   Timer
 *-------------------------------------------------------------------------*/

__sfr __no_init volatile unsigned char TL2 @ 0xCC; /* Timer 2 Low byte */
__sfr __no_init volatile union
{
  unsigned char T2MOD; /* Timer 2 Mode */
  struct /* Timer 2 Mode */
  {
    unsigned char DCEN : 1;
    unsigned char  : 1;
    unsigned char  : 1;
    unsigned char  : 1;
    unsigned char  : 1;
    unsigned char  : 1;
    unsigned char  : 1;
    unsigned char  : 1;
  } T2MOD_bit;
} @ 0xC9;
__sfr __no_init volatile unsigned char TH0 @ 0x8C; /* Timer 0 High */
__sfr __no_init volatile union
{
  unsigned char TCON; /* Timer / Cntr Control */
  struct /* Timer / Cntr Control */
  {
    unsigned char IT0 : 1;
    unsigned char IE0 : 1;
    unsigned char IT1 : 1;
    unsigned char IE1 : 1;
    unsigned char TR0 : 1;
    unsigned char TF0 : 1;
    unsigned char TR1 : 1;
    unsigned char TF1 : 1;
  } TCON_bit;
} @ 0x88;
__sfr __no_init volatile unsigned char TH2 @ 0xCD; /* Timer 2 High byte */
__sfr __no_init volatile unsigned char TH1 @ 0x8D; /* Timer 1 High */
__sfr __no_init volatile union
{
  unsigned char TMOD; /* Timer / Cntr Mode Control */
  struct /* Timer / Cntr Mode Control */
  {
    unsigned char M00 : 1;
    unsigned char M10 : 1;
    unsigned char C_T0 : 1;
    unsigned char Gate0 : 1;
    unsigned char M01 : 1;
    unsigned char M11 : 1;
    unsigned char C_T1 : 1;
    unsigned char Gate1 : 1;
  } TMOD_bit;
} @ 0x89;
__sfr __no_init volatile unsigned char RCAP2L @ 0xCA; /* Timer 2 Reload low */
__sfr __no_init volatile unsigned char RCAP2H @ 0xCB; /* Timer 2 Reload High */
__sfr __no_init volatile unsigned char TL0 @ 0x8A; /* Timer 0 Low */
__sfr __no_init volatile union
{
  unsigned char T2CON; /* Timer 2 Control */
  struct /* Timer 2 Control */
  {
    unsigned char CP_RL2 : 1;
    unsigned char C_T2 : 1;
    unsigned char TR2 : 1;
    unsigned char EXEN2 : 1;
    unsigned char TCLK : 1;
    unsigned char RCLK : 1;
    unsigned char EXF2 : 1;
    unsigned char TF2 : 1;
  } T2CON_bit;
} @ 0xC8;
__sfr __no_init volatile unsigned char TL1 @ 0x8B; /* Timer 1 Low */

/*-------------------------------------------------------------------------
 *   Analog to Digital Converter (ADC)
 *-------------------------------------------------------------------------*/

__sfr __no_init volatile unsigned char ASCL @ 0x95; /* 8-bit Prescaler for ADC clock */
__sfr __no_init volatile union
{
  unsigned char ADAT; /* ADC Data Register */
  struct /* ADC Data Register */
  {
    unsigned char ADAT0 : 1;
    unsigned char ADAT1 : 1;
    unsigned char ADAT2 : 1;
    unsigned char ADAT3 : 1;
    unsigned char ADAT4 : 1;
    unsigned char ADAT5 : 1;
    unsigned char ADAT6 : 1;
    unsigned char ADAT7 : 1;
  } ADAT_bit;
} @ 0x96;
__sfr __no_init volatile union
{
  unsigned char ACON; /* ADC Control Register */
  struct /* ADC Control Register */
  {
    unsigned char ADSF : 1;
    unsigned char ADST : 1;
    unsigned char ADS0 : 1;
    unsigned char ADS1 : 1;
    unsigned char  : 1;
    unsigned char ADEN : 1;
    unsigned char  : 1;
    unsigned char  : 1;
  } ACON_bit;
} @ 0x97;
/*
 * Interrupt Vectors
 */
#define extern0 0x03 /* External interrupt 0 */
#define IE0_int 0x03 /* External interrupt 0 */
#define timer0 0x0B /* Timer 0 Interrupt */
#define TF0_int 0x0B /* Timer 0 Interrupt */
#define extern1 0x13 /* External interrupt 1 */
#define IE1_int 0x13 /* External interrupt 1 */
#define timer1 0x1B /* Timer 1 Interrupt */
#define TF1_int 0x1B /* Timer 1 Interrupt */
#define sio_ti 0x23 /* Serial Port Interrupt */
#define sio_ri 0x23 /* Serial Port Interrupt */
#define TI_int 0x23 /* Serial Port Interrupt */
#define RI_int 0x23 /* Serial Port Interrupt */

#pragma language=default
#endif  /* __IAR_SYSTEMS_ICC__  */

/***************************************************************************
 *   Assembler definitions
 *
 *   The following SFRs are built in in the a8051.exe and can not be
 *   defined explicitly in this file:
 *     ACC,B,PSW,SP,DPL,DPH
 *   The PSW-bits are built-in in the a8051.exe
 *     OV,AC,F0,RS0,RS1,P
 *
 ***************************************************************************/

#ifdef __IAR_SYSTEMS_ASM__


/*-------------------------------------------------------------------------
 *   Interrupt
 *-------------------------------------------------------------------------*/

IEA DEFINE 0xA7 /* Interrupt Enable (2nd) */
IE DEFINE 0xA8 /* Interrupt Enable */
IE_EX0 DEFINE 0xA8.0
IE_ET0 DEFINE 0xA8.1
IE_EX1 DEFINE 0xA8.2
IE_ET1 DEFINE 0xA8.3
IE_ES DEFINE 0xA8.4
IE_ET2 DEFINE 0xA8.5
IE_EA DEFINE 0xA8.7
IPA DEFINE 0xB7 /* Interrupt Priority (2nd) */
IP DEFINE 0xB8 /* Interrupt Priority */
IP_PX0 DEFINE 0xB8.0
IP_PT0 DEFINE 0xB8.1
IP_PX1 DEFINE 0xB8.2
IP_PT1 DEFINE 0xB8.3
IP_PS DEFINE 0xB8.4
IP_PT2 DEFINE 0xB8.5

/*-------------------------------------------------------------------------
 *   I2C Bus
 *-------------------------------------------------------------------------*/

S1SETUP DEFINE 0xD1 /* DDC I2C (S1) Setup */
S2SETUP DEFINE 0xD2 /* I2C (S2) Setup */
S1CON DEFINE 0xD8 /* DDC I2C Control Reg */
S1CON_CR0 DEFINE 0xD8.0
S1CON_CR1 DEFINE 0xD8.1
S1CON_AA DEFINE 0xD8.2
S1CON_ADDR DEFINE 0xD8.3
S1CON_STO DEFINE 0xD8.4
S1CON_STA DEFINE 0xD8.5
S1CON_ENI1 DEFINE 0xD8.6
S1CON_CR2 DEFINE 0xD8.7
S1STA DEFINE 0xD9 /* DDC I2C Status */
S2CON DEFINE 0xDC /* I2C Bus Control Reg */
S2STA DEFINE 0xDD /* I2C Bus Status */
S2ADR DEFINE 0xDF /* I2C address */

/*-------------------------------------------------------------------------
 *   USB
 *-------------------------------------------------------------------------*/

USCL DEFINE 0xE1 /* 8-bit Prescaler for USB logic */
UDT1 DEFINE 0xE6 /* USB Endpt1 Data Xmit */
UDT0 DEFINE 0xE7 /* USB Endpt0 Data Xmit */
UISTA DEFINE 0xE8 /* USB Interrupt Status */
UISTA_RESUMF DEFINE 0xE8.0
UISTA_EOPF DEFINE 0xE8.1
UISTA_RXD1F DEFINE 0xE8.2
UISTA_RXD0F DEFINE 0xE8.3
UISTA_TXD0F DEFINE 0xE8.4
UISTA_RSTF DEFINE 0xE8.5
UISTA_SUSPND DEFINE 0xE8.7
UIEN DEFINE 0xE9 /* USB Interrupt Enable */
UCON0 DEFINE 0xEA /* USB Endpt0 Xmit Control */
UCON1 DEFINE 0xEB /* USB Endpt1 Xmit Control */
UCON2 DEFINE 0xEC /* USB Control Register */
USTA DEFINE 0xED /* USB Endpt0 Status */
UADR DEFINE 0xEE /* USB Address Register */
UDR0 DEFINE 0xEF /* USB Endpt0 Data Recv */

/*-------------------------------------------------------------------------
 *   I/O Port
 *-------------------------------------------------------------------------*/

P0 DEFINE 0x80 /* Port 0 */
P0_P00 DEFINE 0x80.0
P0_P01 DEFINE 0x80.1
P0_P02 DEFINE 0x80.2
P0_P03 DEFINE 0x80.3
P0_P04 DEFINE 0x80.4
P0_P05 DEFINE 0x80.5
P0_P06 DEFINE 0x80.6
P0_P07 DEFINE 0x80.7
P1 DEFINE 0x90 /* Port 1 */
P1_P10 DEFINE 0x90.0
P1_P11 DEFINE 0x90.1
P1_P12 DEFINE 0x90.2
P1_P13 DEFINE 0x90.3
P1_P14 DEFINE 0x90.4
P1_P15 DEFINE 0x90.5
P1_P16 DEFINE 0x90.6
P1_P17 DEFINE 0x90.7
P1SFS DEFINE 0x91 /* Port 1 Select Register */
P3SFS DEFINE 0x93 /* Port 3 Select Register */
P4SFS DEFINE 0x94 /* Port 4 Select Register */
P2 DEFINE 0xA0 /* Port 2 */
P2_P20 DEFINE 0xA0.0
P2_P21 DEFINE 0xA0.1
P2_P22 DEFINE 0xA0.2
P2_P23 DEFINE 0xA0.3
P2_P24 DEFINE 0xA0.4
P2_P25 DEFINE 0xA0.5
P2_P26 DEFINE 0xA0.6
P2_P27 DEFINE 0xA0.7
P3 DEFINE 0xB0 /* Port 3 */
P3_P30 DEFINE 0xB0.0
P3_P31 DEFINE 0xB0.1
P3_P32 DEFINE 0xB0.2
P3_P33 DEFINE 0xB0.3
P3_P34 DEFINE 0xB0.4
P3_P35 DEFINE 0xB0.5
P3_P36 DEFINE 0xB0.6
P3_P37 DEFINE 0xB0.7
P4 DEFINE 0xC0 /* New Port 4 */
P4_P40 DEFINE 0xC0.0
P4_P41 DEFINE 0xC0.1
P4_P42 DEFINE 0xC0.2
P4_P43 DEFINE 0xC0.3
P4_P44 DEFINE 0xC0.4
P4_P45 DEFINE 0xC0.5
P4_P46 DEFINE 0xC0.6
P4_P47 DEFINE 0xC0.7

/*-------------------------------------------------------------------------
 *   PWM
 *-------------------------------------------------------------------------*/

PWMCON DEFINE 0xA1 /* PWM Control Polarity */
PWM1 DEFINE 0xA3 /* PWM1 Output Duty Cycle */
PWM2 DEFINE 0xA4 /* PWM2 Output Duty Cycle */
PWM3 DEFINE 0xA5 /* PWM3 Output Duty Cycle */
PWM4P DEFINE 0xAA /* PWM 4 Period */
PWM4W DEFINE 0xAB /* PWM 4 Pulse Width */

/*-------------------------------------------------------------------------
 *   Serial I/O
 *-------------------------------------------------------------------------*/

SCON DEFINE 0x98 /* Serial Control Register */
SCON_RI DEFINE 0x98.0
SCON_TI DEFINE 0x98.1
SCON_RB8 DEFINE 0x98.2
SCON_TB8 DEFINE 0x98.3
SCON_REN DEFINE 0x98.4
SCON_SM2 DEFINE 0x98.5
SCON_SM1 DEFINE 0x98.6
SCON_SM0 DEFINE 0x98.7
SBUF DEFINE 0x99 /* Serial Buffer */
SCON2 DEFINE 0x9A /* 2nd UART Ctrl Register */
SBUF2 DEFINE 0x9B /* 2nd UART Serial Buffer */

/*-------------------------------------------------------------------------
 *   Watchdog timer
 *-------------------------------------------------------------------------*/

WDRST DEFINE 0xA6 /* Watch Dog Reset */
WDKEY DEFINE 0xAE /* Watch Dog Key Register */

/*-------------------------------------------------------------------------
 *   System Management
 *-------------------------------------------------------------------------*/

PCON DEFINE 0x87 /* Power Ctrl */
PWM0 DEFINE 0xA2 /* Output Duty Cycle */
PSCL0L DEFINE 0xB1 /* Prescaler 0 Low (8-bit) */
PSCL0H DEFINE 0xB2 /* Prescaler 0 High (8-bit) */
PSCL1L DEFINE 0xB3 /* Prescaler 1 Low (8-bit) */
PSCL1H DEFINE 0xB4 /* Prescaler 1 High (8-bit) */
RAMBUF DEFINE 0xD4 /* DDC Ram Buffer */
DDCDAT DEFINE 0xD5 /* DDC Data xmit register */
DDCADR DEFINE 0xD6 /* Addr pointer register */
DDCCON DEFINE 0xD7 /* DDC Control Register */
S1DAT DEFINE 0xDA /* Data Hold Register */
S1ADR DEFINE 0xDB /* DDC I2C address */
S2DAT DEFINE 0xDE /* Data Hold Register */

/*-------------------------------------------------------------------------
 *   Timer
 *-------------------------------------------------------------------------*/

TCON DEFINE 0x88 /* Timer / Cntr Control */
TCON_IT0 DEFINE 0x88.0
TCON_IE0 DEFINE 0x88.1
TCON_IT1 DEFINE 0x88.2
TCON_IE1 DEFINE 0x88.3
TCON_TR0 DEFINE 0x88.4
TCON_TF0 DEFINE 0x88.5
TCON_TR1 DEFINE 0x88.6
TCON_TF1 DEFINE 0x88.7
TMOD DEFINE 0x89 /* Timer / Cntr Mode Control */
TL0 DEFINE 0x8A /* Timer 0 Low */
TL1 DEFINE 0x8B /* Timer 1 Low */
TH0 DEFINE 0x8C /* Timer 0 High */
TH1 DEFINE 0x8D /* Timer 1 High */
T2CON DEFINE 0xC8 /* Timer 2 Control */
T2CON_CP_RL2 DEFINE 0xC8.0
T2CON_C_T2 DEFINE 0xC8.1
T2CON_TR2 DEFINE 0xC8.2
T2CON_EXEN2 DEFINE 0xC8.3
T2CON_TCLK DEFINE 0xC8.4
T2CON_RCLK DEFINE 0xC8.5
T2CON_EXF2 DEFINE 0xC8.6
T2CON_TF2 DEFINE 0xC8.7
T2MOD DEFINE 0xC9 /* Timer 2 Mode */
RCAP2L DEFINE 0xCA /* Timer 2 Reload low */
RCAP2H DEFINE 0xCB /* Timer 2 Reload High */
TL2 DEFINE 0xCC /* Timer 2 Low byte */
TH2 DEFINE 0xCD /* Timer 2 High byte */

/*-------------------------------------------------------------------------
 *   Analog to Digital Converter (ADC)
 *-------------------------------------------------------------------------*/

ASCL DEFINE 0x95 /* 8-bit Prescaler for ADC clock */
ADAT DEFINE 0x96 /* ADC Data Register */
ACON DEFINE 0x97 /* ADC Control Register */

#endif /* __IAR_SYSTEMS_ASM__*/
#endif /* IOUPSD3254A_H */
// **************************Fichier.h
// Auteur:       Vincent Chouinard
// Date:         20 juillet 2014
// Version:      1.0
// Modification: Aucune
//
// Compilateur:  IAR 8.1
//
// Description:
// *****************************************************************************
#include "_DeclarationGenerale.h" // Raccourcis de programmation & variables
#include "_DeclarationGenerale.h" // Raccourcis Linguistiques utiles
#include "ConversionKeilToIAR.h"  //
#include "CLClavier.h"            // Pour utiliser le clavier
#include "CLEcran.h"              // Pour utiliser l'écran
#include "stdio.h"                // Pour faire des printf

//LinuxLinuxLinuxLinuxLinuxLinuxLinuxLinuxLinuxLinuxLinuxLinuxLinuxLinuxLinuxTUX// **************************TypeCible.h
// Auteur:      Vincent Chouinard
// Date:        4 fevrier 2014
// Version :    1.0
//
// Compilateur: Keil 4.72.9 & IAR 8.1
// Description: Fichier pour choisir le compilateur
//
// *****************************************************************************
#ifndef TYPECIBLEH
  #define TYPECIBLEH

// *****************************************************************************
//           CHOIX DU COMPILATEUR
// *****************************************************************************
  //#define DALLAS89C450             // Uncomment to compile for Dallas DS89C450
  //#define UPSD3254A                // Uncomment to compile for Dallas UPSD3254A

// *****************************************************************************
//           DEFINITION DES REGISTRES DU CPU CHOISI
// *****************************************************************************
#ifdef DALLAS89C450
   #include "ioDS89C450.h"  // I/O access for DS89C450
#endif

#ifdef UPSD3254A
   #include "iouPSD3254A.h" // I/O access for UPSD3254A
#endif

#endif //TYPECIBLEH
//LinuxLinuxLinuxLinuxLinuxLinuxLinuxLinuxLinuxLinuxLinuxLinuxLinuxLinuxLinuxTUX// ************************** FICHIER: ConversionKeilToIAR.h
//
//  DEFINITION POUR CONVERTIR DU COMPILATEUR KEIL
//  VERS LE COMPILATEUR IAR
//
//  Application realisee avec IAR C++ 8.10
//
//  Definition des fonctions de la classe CLMaison.
//
//  AUTEUR : DANIEL BRETON
//  DATE CREATION :    2012-12-20       VERSION: 1.0
//  DATE MODIFICATION: ****-**-**
//
// ************************************************************************

#include "_TypeCible.h"

#ifndef CONVERSIONKEILTOIARH
  #define CONVERSIONKEILTOIARH

// *****************************************************************************
//
// DEFINITION DES  MODELES DE MEMOIRE
//
// *****************************************************************************

#define data  __data
#define xdata __xdata
#define code  __code


// *****************************************************************************
#ifdef DALLAS89C450

// Interruption activation
  #define EA            IE_bit.EA
  #define ES            IE_bit.ES0
  #define ES2           IE_bit.ES1
  #define EX0           IE_bit.EX0

// Interruption priorite
  #define LPX0          IP0_bit.LPX0
  #define MPX0          IP1_bit.MPX0

  #define LPS           IP0_bit.LPS0
  #define MPS           IP1_bit.MPS0
  #define LPS2          IP0_bit.LPS1
  #define MPS2          IP1_bit.MPS1

// Interruption niveau activite.
  #define IT0           TCON_bit.IT0
  #define TR1           TCON_bit.TR1

  #define TCLK2         T2CON_bit.TCKL
  #define TRCLK2        T2CON_bit.RCLK
  #define TR2           T2CON_bit.TR2

  #define T2MOD_DCEN    T2MOD_bit.DCEN
//
  #define SMOD          PCON_bit.SMOD_0
  #define SMOD2         WDCON_bit.SMOD_1

  #define SCON          SCON0
  #define SBUF          SBUF0
  #define RI_0          SCON0_bit.RI_0
  #define TI_0          SCON0_bit.TI_0

  #define SCON2         SCON1
  #define SBUF2         SBUF1
  #define RI_2          SCON1_bit.RI_1
  #define TI_2          SCON1_bit.TI_1

  #define SM0_0         SCON0_bit.SM0_FE_0
  #define SM1_0         SCON0_bit.SM1_0
  #define SM2_0         SCON0_bit.SM2_0
  #define REN_0         SCON0_bit.REN_0

  #define SM0_2         SCON1_bit.SM0_FE_1
  #define SM1_2         SCON1_bit.SM1_1
  #define SM2_2         SCON1_bit.SM2_1
  #define REN_2         SCON1_bit.REN_1


  #define P1_0          P1_bit.P10_T2          // Port 1.0
  #define P1_1          P1_bit.P11_T2EX        // Port 1.1
  #define P1_2          P1_bit.P12_RXD1        // Port 1.2
  #define P1_3          P1_bit.P13_TXD1        // Port 1.3
  #define P1_4          P1_bit.P14_INT2        // Port 1.4
  #define P1_5          P1_bit.P15_INT3        // Port 1.5
  #define P1_6          P1_bit.P16_INT4        // Port 1.6
  #define P1_7          P1_bit.P17_INT5        // Port 1.7


  #define P3_0          P3_bit.P30_RXD0        // Port 3.0
  #define P3_1          P3_bit.P31_TXD0        // Port 3.1
  #define P3_2          P3_bit.P32_INT0        // Port 3.2
  #define P3_3          P3_bit.P33_INT1        // Port 3.3
  #define P3_4          P3_bit.P34_T0          // Port 3.4
  #define P3_5          P3_bit.P35_T1          // Port 3.5
  #define P3_6          P3_bit.P36_WR          // Port 3.6
  #define P3_7          P3_bit.P37_RD          // Port 3.7



#endif // DALLAS89C450
// *****************************************************************************

// *****************************************************************************
#ifdef UPSD3254A
// Interruption activation
  #define EA            IE_bit.EA
  #define ES            IE_bit.ES
  #define ES2           IEA_bit.ES2
  #define EX0           IE_bit.EX0

// Interruption priorite
  #define PX0           IP_bit.PX0
  #define PS            IP_bit.PS
  #define PS2           IPA_bit.PS2

// Interruption niveau activite.
  #define IT0           TCON_bit.IT0
  #define TR1           TCON_bit.TR1

  #define TCLK2         T2CON_bit.TCKL
  #define TRCLK2        T2CON_bit.RCLK
  #define TR2           T2CON_bit.TR2

  #define T2MOD_DCEN    T2MOD_bit.DCEN


  #define SMOD          PCON_bit.SMOD
  #define SMOD2         PCON_bit.SMOD1


 // #define ES0 IE_bit.ES

  #define RI_0          SCON_bit.RI
  #define TI_0          SCON_bit.TI
  #define SM0_0         SCON_bit.SM0
  #define SM1_0         SCON_bit.SM1
  #define SM2_0         SCON_bit.SM2
  #define REN_0         SCON_bit.REN

  #define TI_2          SCON2_bit.TI
  #define RI_2          SCON2_bit.RI
  #define SM0_2         SCON2_bit.SM0
  #define SM1_2         SCON2_bit.SM1
  #define SM2_2         SCON2_bit.SM2
  #define REN_2         SCON2_bit.REN

  #define WATCHDOG_OFF WDKEY = 0x55; // Disable watchdog.

  #define P1_0          P1_bit.P10        // Port 1.0
  #define P1_1          P1_bit.P11        // Port 1.1
  #define P1_2          P1_bit.P12        // Port 1.2
  #define P1_3          P1_bit.P13        // Port 1.3
  #define P1_4          P1_bit.P14        // Port 1.4
  #define P1_5          P1_bit.P15        // Port 1.5
  #define P1_6          P1_bit.P16        // Port 1.6
  #define P1_7          P1_bit.P17        // Port 1.7

  #define P3_0          P3_bit.P30        // Port 3.0
  #define P3_1          P3_bit.P31        // Port 3.1
  #define P3_2          P3_bit.P32        // Port 3.2
  #define P3_3          P3_bit.P33        // Port 3.3
  #define P3_4          P3_bit.P34        // Port 3.4
  #define P3_5          P3_bit.P35        // Port 3.5
  #define P3_6          P3_bit.P36        // Port 3.6
  #define P3_7          P3_bit.P37        // Port 3.7

  #define P4_0          P4_bit.P40        // Port 4.0
  #define P4_1          P4_bit.P41        // Port 4.1
  #define P4_2          P4_bit.P42        // Port 4.2
  #define P4_3          P4_bit.P43        // Port 4.3
  #define P4_4          P4_bit.P44        // Port 4.4
  #define P4_5          P4_bit.P45        // Port 4.5
  #define P4_6          P4_bit.P46        // Port 4.6
  #define P4_7          P4_bit.P47        // Port 4.7


#endif // UPSD3254A
// *****************************************************************************

#endif // CONVERSIONKEILTOIARH
// ***************** FICHIER: _DeclarationGenerale.h
//
//  DEFINITION GENERALE POUR UN PROJET
//
//  Application realisee avec Keil 4.72.9 && IAR 8.1
//
//  AUTEUR : DANIEL BRETON
//  DATE CREATION :    4 septembre 2013      VERSION: 1.2
//  DATE MODIFICATION: 23 janvier 2014
//                        ---> Ajout de TypeDEF
//                     6 FEVRIER 2014
//                        ---> Ajout de structures & de headers pour IAR 8.1
// ****************************************************************************
//                           LES INCLUDES
// ****************************************************************************
#include "ConversionKeilToIAR.h"     // Conversion KEIL <--> IAR
#include "TypeCible.h"               // Selection du CPU

#ifndef DECLARATIONGENERALEH
  #define DECLARATIONGENERALEH
// ****************************************************************************
//                            DEFINITION DE MOTS CLES
// ****************************************************************************
#define ACTIF   1
#define INACTIF 0

#define ON      1
#define OFF     0

#define VRAI    1
#define FAUX    0

#define HIGH    1
#define LOW     0

#define ACK     0
#define NOACK   1

#define TRUE    1
#define FALSE   0

#define COMPLET   1
#define INCOMPLET 0

// ****************************************************************************
//                            VARIABLES SHORTCUTS
// ****************************************************************************
#ifndef NULL
  #define NULL 0
#endif

typedef float              F;
typedef signed long        L;
typedef signed char        C;
typedef unsigned  long     UL;
typedef unsigned char      UC;
typedef unsigned int       UI;
typedef signed short int   SI;
typedef unsigned short int USI;
typedef signed char        INT8;
typedef unsigned char      UINT8;
typedef unsigned short int UINT16;
typedef signed long        INT32;
typedef unsigned long int  UINT32;
typedef void               VOID;

#define BARGRAPH1 0xFA00  // Adresses des composantes sur la carte binaire.
#define BARGRAPH2 0xFB00
#define DIPSW1    0xFA00
#define DIPSW2    0xFB00
#define SEGMENT1  0xFD00
#define SEGMENT2  0xFC00

// ****************************************************************************
//                            DEFINITION DES STRUCTURES
// ****************************************************************************
struct STTemps
 {
   UC ucSeconde;
   UC ucMinute;
   UC ucHeure;
 };

struct STChampBit
 {
   UC bBit0 : 1;
   UC bBit1 : 1;
   UC bBit2 : 1;
   UC bBit3 : 1;
   UC bBit4 : 1;
   UC bBit5 : 1;
   UC bBit6 : 1;
   UC bBit7 : 1;
 };

struct STMot
 {
   UC ucPetitMot : 4;
   UC ucGrandMot : 4;
 };

union UNOctetBit
 {
   struct STMot      stMot;
   struct STChampBit stChampBit;
   UC ucOctet;
 };

struct STDoubleOctet
 {
   UC ucOctet1;
   UC ucOctet2;
 };

union UNEntierOctet
 {
   struct STDoubleOctet stDoubleOctet;
   USI uiEntier;
 };

struct STQuadOctet
 {
   UC ucOctet1;
   UC ucOctet2;
   UC ucOctet3;
   UC ucOctet4;
 };

union UNLongOctet
 {
   struct STQuadOctet stQuadOctet;
   UL ulLong;
 };
#endif
//LinuxLinuxLinuxLinuxLinuxLinuxLinuxLinuxLinuxLinuxLinuxLinuxLinuxLinuxLinuxTUX


// ************************** FICHIER: _TYPECIBLE.H
//
//  Application realisee avec IAR C++ 8.10
//
//  Fichier pour selectionner la CIBLE et pour inclure
//  le fichier de conversion des définitions des registres SFR.
//
//  AUTEUR : DANIEL BRETON
//  DATE CREATION :    2013-01-23       VERSION: 1.0
//  DATE MODIFICATION: ****-**-**
// ************************************************************************
#ifndef TYPECIBLEH
  #define TYPECIBLEH

#define DALLAS89C450
//  #define UPSD3254A

// ********************************************************************
//  Fichier de definition des registres en fonction de la CIBLE.
//
  #ifdef DALLAS89C450
    #include <ioDS89C450.h>
  #endif


  #ifdef UPSD3254A
    #include <ioUPSD3254A.h>
  #endif
//
// ********************************************************************
#endif
\end{lstlisting}







\end{document}

