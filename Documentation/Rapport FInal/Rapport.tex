\documentclass[10pt,a4paper,final]{article}

\usepackage{geometry}                            
\geometry{a4paper,total={210mm,297mm},left=10mm,right=10mm,top=20mm,bottom=20mm} 

\usepackage{fancyhdr}          % Met les pieds et têtes de pages
\pagestyle{fancy}

\usepackage[francais]{babel}   % Met les accents français
\usepackage[utf8]{inputenc}
\usepackage[pdftex]{graphicx}

\usepackage{graphicx}
\usepackage{framed}
\usepackage{tabularx}          % Liste les tableaux en table des matières
\usepackage{listings}
\usepackage{lipsum}
\usepackage{multicol}          % Permet le multicolone
\usepackage{xcolor}

\renewcommand\familydefault{\sfdefault}

\usepackage{tgheros}                
\usepackage[defaultmono]{droidmono} 
\usepackage{amsmath,amssymb,amsthm,textcomp}
%\usepackage{ubuntu}
   % Packages et dépendances nécessaires
\include{TeX/SourceCode} % Permet de mettre du code source dans le document

\begin{document}
\renewcommand\headrulewidth{0pt}
\fancyfoot[C]{ }






\begin{center}
\includegraphics[scale=1]{Figures/School_Logo.jpg}~\\[1cm]  
\textsc{\LARGE \'{E}lectronique Programmable et Robotique}\\[1.5cm]
\Large 247-6[1-2-3-4]7-LI\\[0.5cm]
{ \huge \bfseries Projet de 5$\mathbf{^{e}}$ session \\[0.4cm] }
\HRule \\[1.5cm]





\begin{multicols}{2}
\begin{flushleft}



\textbf{Étudiants:}\\

\bigskip

Vincent Chouinard\\
Hicham Safoine\\
Gabriel Fortin-Bélanger\\ 
Louis-Nomand Ang-Houle\\




\end{flushleft}
\vfill
\begin{flushright}

\textbf{Professeurs:}\\
\medskip
Ali Tadli\\
Alain Champagne\\
Stéphane Deschênes\\
Étienne Tremblay\\



\end{flushright}
\end{multicols}

\bigskip
\bigskip


\includegraphics[scale=0.6]{Figures/Picture_for_Title.jpg} 

\vfill
L'usine à gaz, et le gaz, c'est de l'air!
\bigskip

{\large \today}
\end{center}






\pagebreak
\begin{spacing}{1.24999}
\tableofcontents 
%\pagebreak
\listoffigures  
\listoftables 
\end{spacing}  
\pagebreak






\renewcommand\headrulewidth{1pt}
\fancyhead[L]{247-6[1-2-3-4]7-LI}
\fancyhead[R]{Projet de 5$\mathbf{^{e}}$ session}

\renewcommand\footrulewidth{1pt}
\fancyfoot[C]{\textbf{page \thepage}}
\fancyfoot[R]{\today}
\fancyfoot[L]{V.C., H.S., G.F-B., L-N.A-H.}
\begin{spacing}{1.5}

\section{Présentation du projet}

\subsection{Explication du projet}

\subsection{Schéma bloc du système}
\subsubsection{Bloc 1}

\subsubsection{Bloc 2}

\subsubsection{Bloc 3}

\subsubsection{Bloc 4}

\subsection{Liste des logiciels}
\begin{flushleft}
\HRule
\end{flushleft}
\begin{flushleft}

\begin{multicols}{3}

\textbf{Terminaux}
\begin{itemize}
\item[•]UART Master 1.0.3
\item[•]Serializ3r 1.0.2
\item[•]TerraTerm
\item[•]Putty
\item[•]GTKterm 0.99.7-rc1
\item[•]xTerminator
\item[•]CAPS
\item[•]tinyBootloader
\end{itemize}

\textbf{Gestion du projet}
\begin{itemize}
\item[•]MS Project 2012
\item[•]Git Hub
\end{itemize}

\textbf{Compilateurs et IDE}
\begin{itemize}
\item[•]Visual Studio 2013
\item[•]Visual Studio 2010
\item[•]IAR 8.20
\item[•]MPLAB
\end{itemize}

\textbf{Éditeur de texte}
\begin{itemize}
\item[•]Notepad++
\item[•]gedit
\item[•]medit 1.2.0
\end{itemize}
\textbf{Schémas électriques}

\begin{itemize}
\item[•]OrCAD 16.2
\end{itemize}

\bigskip

\textbf{Système d'exploitation}
\begin{itemize}
\item[•]Windows 7 SP1
\item[•]Windows 8.1
\item[•]Windows XP SP3
\item[•]Fedora 20
\item[•]CentOS
\item[•]Lubuntu 14.10
\end{itemize}

\textbf{Autres}
\begin{itemize}
\item[•]VMWare Workstation 10
\item[•]TeXmaker 4.3
\item[•]Dukto R6
\item[•]Dia
\item[•]Festo configuration tool
\end{itemize}

\end{multicols}
\end{flushleft}
\begin{flushleft}
\HRule
\end{flushleft}

\pagebreak
\subsection{Liste des trames}

\begin{table}[!ht]
\caption{Index des identifianst matériel CAN}
\medskip
\centering
\begin{tabular}{|l|c|}
\hline 
\textbf{Device} & \textbf{ID matériel} \\ 
\hline 
Ordinateur & 000 \\ 
\hline 
SOC8200 & 001 \\ 
\hline 
Station 1 & 002 \\ 
\hline 
Station 2  & 003 \\ 
\hline 
Véhicule  & 004 \\ 
\hline 
\end{tabular} 
\label{tab:testtab1}
\end{table} 


\begin{table}[!ht]
\caption{Index des trames CAN}
\medskip
\centering
\begin{tabular}{|l|c|l|c|}
\hline 
\textbf{Fonctionnalité} & \textbf{Composante} & \textbf{Données} & \textbf{TimeStamp}\\
\hline 
Démarre le véhcule & 0x00 & 0x00 & TimeStamp\\ 
\hline 
Arrête le véhicule & 0x00 & 0x01 & TimeStamp\\ 
\hline 
Le véhicule est arrêté & 0x01 & 0x00 & TimeStamp\\ 
\hline 
Le véhicule est en marche& 0x01 & 0x01 & TimeStamp\\ 
\hline 
Le véhicule est hors circuit & 0x01 & 0x02 & TimeStamp\\ 
\hline 
Vitesse (0-100) & 0x02 & 0x00 à 0x64 & TimeStamp\\ 
\hline 
Battrie & 0x03 & 0x00 à 0x64 & TimeStamp\\ 
\hline 
Couleur du bloc & 0x04 & 0x00 à 0x02 & TimeStamp\\ 
\hline 
Poids du bloc & 0x05 & 0x00 à 0x64 & TimeStamp\\ 
\hline 
Envoyer l'heure & 0x06 & à déterminer  & TimeStamp\\ 
\hline 
No. de la station & 0x07 & 0x00 à 0x02 & TimeStamp\\ 
\hline 
Demande de l'historique & 0xC0 & 0x00 & TimeStamp\\ 
\hline 
Direction horaire et antihoraire & 0x08 & 0x00 à 0x01 & TimeStamp\\
\hline
\end{tabular} 
\label{tab:testtab1}
\end{table}

\begin{table}[!ht]
\caption{Index des communications CAN}
\medskip
\centering
\begin{tabular}{|l|l|c|l|c|l|c|}
\hline 
\textbf{Émetteur} & \textbf{Action} & \textbf{ID receveur} & \textbf{Donnée envoyée} & \textbf{TimeStamp} & \textbf{Récepteur} & \textbf{Erreur}\\ 
\hline 
Ordinateur & Démarrer le véhicule & 004 & 00 00 & TimeStamp & Véhicule &  F1\\ 
\hline 
Ordinateur & Arrêter le véhicule & 004 & 00 01 & TimeStamp & Véhicule &  F2\\ 
\hline 
Véhicule & Dit: je suis arrêté & 000 & 01 00  & TimeStamp & Ordinateur &  F3\\
\hline 
Véhicule & Dit: j'avance & 000 & 01 01 & TimeStamp & Ordinateur  &  F4\\  
\hline 
Véhicule & Dit: je suis hors circuit & 000 & 01 02 & TimeStamp & Ordinateur &  F5\\ 
\hline 
Véhicule & Dit sa vitesse & 000 & 02 [00 à 64] & TimeStamp & Ordinateur &  F6\\ 
\hline 
Véhicule & Dit le niveau de sa battrie & 000 & 03 [00 à 64] & TimeStamp & Ordinateur &  F7\\ 
\hline 
Station 1 & Dit bloc = métal & 000 & 04 00 & TimeStamp & Ordinateur &  F8\\ 
\hline  
Station 1 & Dit bloc = orange & 000 & 04 01 & TimeStamp & Ordinateur &  F9\\
\hline 
Station 1 & Dit bloc = noir & 000 & 04 02 & TimeStamp & Ordinateur &  FA\\
\hline 
Station 1 & Dit le poid du bloc & 000 & 05 [00 à 64] & TimeStamp & Ordinateur &  FB\\ 
\hline 
Voiture & Dit qu'elle est à la station 1 & 000 & 07 00 & TimeStamp & Ordinateur &  FC\\ 
\hline 
Voiture & Dit qu'elle est à la station 2 & 000 & 07 01 & TimeStamp & Ordinateur &  FD\\
\hline 
Ordinateur & Envoie l'heure & 003 & 06 à déterminer & TimeStamp & Station 1 &  FE\\ 
\hline 
Ordinateur & Demande le LOG & 001 & C0 00 & TimeStamp & SOC8200 & E0\\ 
\hline
Ordinateur & Exige Horaire & 004 & 08 00 & TimeStamp & Véhicule & E1\\ 
\hline
Ordinateur & Exige Antihoraire & 004 & 08 01 & TimeStamp & Véhicule & E2\\ 
\hline
\end{tabular} 
\label{tab:testtab1}
\end{table}
\begin{flushleft}
\textbf{Note:} Il faut définir les TimeStamps et la checkSUM\\
\textbf{Note:} La sttion no.1 relaie les données entre l'ordinateur et la station no.3 (pesage), entre l'ordinateur et la station no.2 (table festo) via xbee et entre la voiture et le PC via xbee.
\end{flushleft}
\pagebreak
\textbf{Note:} FF, c'est la checkSUM, mais elle n'a pas encore été faite \\
\textbf{Note:} Il faut ajouter le TimeStamp 
\begin{verbatim}
   CAN.SendMCP("0100FF"); // Arrêté
   CAN.SendMCP("0101FF"); // En marche
   CAN.SendMCP("0102FF"); // Hors circuit
   CAN.SendMCP("0264FF"); // Vitesse maximale
   CAN.SendMCP("0364FF"); // Battrie pleine
   CAN.SendMCP("0400FF"); // Bloc métallique
   CAN.SendMCP("0401FF"); // Bloc noire
   CAN.SendMCP("0402FF"); // Bloc orange
   CAN.SendMCP("050064"); // Le bloc est lourd
   CAN.SendMCP("0700FF"); // Rendu à la station 1
   CAN.SendMCP("0701FF"); // Rendu à la station 2
   CAN.SendMCP("0702FF"); // Rendu à la station 3
\end{verbatim}

\pagebreak

\end{spacing}

\textbf{Comment ajouter une entrée à l'historique}
\begin{lstlisting}
Historique.AppendText("\r\n");
Historique.AppendText("\r\n");
Historique.AppendText(DateTime.Now.ToString("yyyy-MM-dd HH:mm:ss"));
Historique.AppendText("\r\n");
Historique.AppendText("Insérer messge ici");   // Enregistre le log de la connexion
\end{lstlisting}

\begin{center}
\HRule
\end{center}

\textbf{Comment envoyer une trame sécurisée}
\begin{lstlisting}
try
{
 TPCANMsg CANMsg;
 TPCANStatus stsResult;

 CANMsg = new TPCANMsg();
 CANMsg.DATA = new byte[8];

 CANMsg.ID = 006; // 006 c'est pour faire des tests. Mettre 004 pour la version finale
 CANMsg.LEN = 4;  // Note: l'index commence à zero, donc 3 = 4
 CANMsg.MSGTYPE = TPCANMessageType.PCAN_MESSAGE_STANDARD;

 for (int i = 0; i < CANMsg.LEN; i++)
    {
    if (i == 0) { CANMsg.DATA[0] = 48; } // Note: le programme .HEX de Gab ne reconnait que l'ASCII
    if (i == 1) { CANMsg.DATA[1] = 48; } // Ce programme en C# envoie en base 10
    if (i == 2) { CANMsg.DATA[2] = 48; } // 48 en base 10 = 0x30 en hexa
    if (i == 3) { CANMsg.DATA[3] = 48; } // 0x30 en hexa = 0 en ascii
    if (i == 4) { CANMsg.DATA[4] = 00; } // Donc, pour écrire un 0 sur la carte Dallas de GAB,
    if (i == 5) { CANMsg.DATA[5] = 00; } // il faut envoyer 48
    if (i == 6) { CANMsg.DATA[6] = 00; }
    if (i == 7) { CANMsg.DATA[7] = 00; }               
   }

 stsResult = PCANBasic.Write(m_PcanHandle, ref CANMsg);

 if (stsResult == TPCANStatus.PCAN_ERROR_OK)
   {
    Historique.AppendText("\r\n");
    Historique.AppendText("\r\n");
    Historique.AppendText(DateTime.Now.ToString("yyyy-MM-dd HH:mm:ss"));
    Historique.AppendText("\r\n");
    Historique.AppendText("XYZ command successfully sent");   
   }
 else
   {
    Historique.AppendText("\r\n");
    Historique.AppendText("\r\n");
    Historique.AppendText(DateTime.Now.ToString("yyyy-MM-dd HH:mm:ss"));
    Historique.AppendText("\r\n");
    Historique.AppendText("Error sending START command");  
   }
 }
catch
   {
    Historique.AppendText("\r\n");
    Historique.AppendText("\r\n");
    Historique.AppendText(DateTime.Now.ToString("yyyy-MM-dd HH:mm:ss"));
    Historique.AppendText("\r\n");
    Historique.AppendText("Error sending data on CAN bus (try catch error)"); 
   }
\end{lstlisting}
\pagebreak
\textbf{Comment prendre une décision en fonction de la trame reçue}
\begin{lstlisting}
ListViewItem lviCurrentItem;
try
  {
   foreach (MessageStatus msgStatus in m_LastMsgsList)
     {
      if (msgStatus.MarkedAsUpdated)
        {
         msgStatus.MarkedAsUpdated = false;
         lviCurrentItem = lstMessages.Items[msgStatus.Position];
         lviCurrentItem.SubItems[2].Text = msgStatus.CANMsg.LEN.ToString();
         lviCurrentItem.SubItems[3].Text = msgStatus.DataString;
         lviCurrentItem.SubItems[4].Text = msgStatus.Count.ToString();
         lviCurrentItem.SubItems[5].Text = msgStatus.TimeString;

         textBox1.Text = lviCurrentItem.SubItems[3].Text;
        }
      }

   if (textBox1.Text == "41 42 43 44 45 46 ") // Si cette trame est reçue
     {
      label19.Text = "BINGO";   // Prend cette décision

      Historique.AppendText("\r\n");
      Historique.AppendText("\r\n");
      Historique.AppendText(DateTime.Now.ToString("yyyy-MM-dd HH:mm:ss"));
      Historique.AppendText("\r\n");
      Historique.AppendText("Donnée reçue"); 
     }
  }
catch
  {
   Historique.AppendText("\r\n");
   Historique.AppendText("\r\n");
   Historique.AppendText(DateTime.Now.ToString("yyyy-MM-dd HH:mm:ss"));
   Historique.AppendText("\r\n");
   Historique.AppendText("Erreur de réception try catch");
  }
\end{lstlisting}
\begin{spacing}{1.5}


\section{Le matériel}
\subsection{Bloc 1}

\subsection{Bloc 2}

\subsection{Bloc 3}

\subsection{Bloc 4}



\subsection{Explication des types de liens}



\subsection{Explication des trames}
\subsubsection{RS-232}

\subsubsection{CAN}

\subsubsection{XBEE}




\subsection{Liste des pièces}
\begin{center}
\HRule
\end{center}
\begin{multicols}{3}
\begin{itemize}
\item[•]Carte Dallas
\item[•]Carte uPSD
\item[•]SOC 8200
\item[•]PIC18Fmachin
\item[•]Carte d'extension SPI
\item[•]Carte d'extension I2C
\item[•]Carte CAN MCP2515
\item[•]XBEE
\item[•]Table FESTO
\item[•]Carte d'extension IO
\item[•]Carte connecteur DAC ADC
\item[•]
\item[•]
\item[•]
\item[•]
\item[•]
\item[•]
\item[•]
\item[•]
\end{itemize}
\end{multicols}
\begin{center}
\HRule
\end{center}


\subsubsection{Liens web}

\subsubsection{Datasheet des PDF}


\pagebreak
\section{Interface PC}

\subsection{Structure du programme}

\subsection{Explication des trames}

\subsection{Ordre de gestion des tâches}





\pagebreak
\section{Logiciel du SOC8200}
\subsection{Description du programme}

\subsection{Schéma bloc}
\subsubsection{Du code}

\subsubsection{Du script shell}

\subsection{Gestion des processus et du temps de CPU}

\subsection{Format et récupration des logs}

\subsection{Liste des tests et logiciels}






\pagebreak
\section{Logiciel de la station 1 et 2 et du bolide}

\subsection{La station no.1}

\subsection{La station no.2}

\subsection{Le bolide}

\subsection{Procédure de compilation sur IAR}

\subsection{Procédure de vérification}




\section{Logiciel du module PIC18F258}
\subsection{Description du fonctionnement du programme}

\subsection{Procédure de compilation sur MPLAB}

\subsection{Procédure de vérification}






\pagebreak
\section{Conclusion}

\subsection{Ce que le projet m'a apporté}
\subsubsection{Vincent Chouinard}

\subsubsection{Hicham Safoine}

\subsubsection{Gabriel Fortin-Bélanger}

\subsubsection{Louis-Norman Ang-Houle}



\subsection{Difficultés et corrections}
\subsubsection{Vincent Chouinard}

\subsubsection{Hicham Safoine}

\subsubsection{Gabriel Fortin-Bélanger}

\subsubsection{Louis-Norman Ang-Houle}


\subsection{Ce que j'ai aimé ou pas}
\subsubsection{Vincent Chouinard}

\subsubsection{Hicham Safoine}

\subsubsection{Gabriel Fortin-Bélanger}

\subsubsection{Louis-Norman Ang-Houle}
\end{spacing}
\end{document}

