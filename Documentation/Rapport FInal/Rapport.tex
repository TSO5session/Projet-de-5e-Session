\documentclass[10pt,a4paper,final]{article}

\usepackage{geometry}                            
\geometry{a4paper,total={210mm,297mm},left=10mm,right=10mm,top=20mm,bottom=20mm} 

\usepackage{fancyhdr}          % Met les pieds et têtes de pages
\pagestyle{fancy}

\usepackage[francais]{babel}   % Met les accents français
\usepackage[utf8]{inputenc}
\usepackage[pdftex]{graphicx}

\usepackage{graphicx}
\usepackage{framed}
\usepackage{tabularx}          % Liste les tableaux en table des matières
\usepackage{listings}
\usepackage{lipsum}
\usepackage{multicol}          % Permet le multicolone
\usepackage{xcolor}

\renewcommand\familydefault{\sfdefault}

\usepackage{tgheros}                
\usepackage[defaultmono]{droidmono} 
\usepackage{amsmath,amssymb,amsthm,textcomp}
%\usepackage{ubuntu}
   % Packages et dépendances nécessaires
\include{TeX/SourceCode} % Permet de mettre du code source dans le document

\begin{document}
\renewcommand\headrulewidth{0pt}
\fancyfoot[C]{ }






\begin{center}
\includegraphics[scale=1]{Figures/School_Logo.jpg}~\\[1cm]  
\textsc{\LARGE \'{E}lectronique Programmable et Robotique}\\[1.5cm]
\Large 247-6[1-2-3-4]7-LI\\[0.5cm]
{ \huge \bfseries Projet de 5$\mathbf{^{e}}$ session \\[0.4cm] }
\HRule \\[1.5cm]





\begin{multicols}{2}
\begin{flushleft}



\textbf{Étudiants:}\\

\bigskip

Vincent Chouinard\\
Hicham Safoine\\
Gabriel Fortin-Bélanger\\ 
Louis-Nomand Ang-Houle\\




\end{flushleft}
\vfill
\begin{flushright}

\textbf{Professeurs:}\\
\medskip
Ali Tadli\\
Alain Champagne\\
Stéphane Deschênes\\
Étienne Tremblay\\



\end{flushright}
\end{multicols}

\bigskip
\bigskip


\includegraphics[scale=0.6]{Figures/Picture_for_Title.jpg} 

\vfill
L'usine à gaz, et le gaz, c'est de l'air!
\bigskip

{\large \today}
\end{center}






\pagebreak
\begin{spacing}{1.5}
\tableofcontents 
%\pagebreak
\listoffigures  
\listoftables 
\end{spacing}  
\pagebreak






\renewcommand\headrulewidth{1pt}
\fancyhead[L]{247-6[1-2-3-4]7-LI}
\fancyhead[R]{Projet de 5$\mathbf{^{e}}$ session}

\renewcommand\footrulewidth{1pt}
\fancyfoot[C]{\textbf{page \thepage}}
\fancyfoot[R]{\today}
\fancyfoot[L]{V.C., H.S., G.F-B., L-N.A-H.}
\begin{spacing}{1.5}

\section{Présentation du projet}

\subsection{Explication du projet}

\subsection{Schéma bloc du système}
\subsubsection{Bloc 1}

\subsubsection{Bloc 2}

\subsubsection{Bloc 3}

\subsubsection{Bloc 4}

\subsection{Liste des logiciels}





\pagebreak
\section{Le matériel}
\subsection{Bloc 1}

\subsection{Bloc 2}

\subsection{Bloc 3}

\subsection{Bloc 4}



\subsection{Explication des types de liens}



\subsection{Explication des trames}
\subsubsection{RS-232}

\subsubsection{CAN}

\subsubsection{XBEE}




\subsection{Liste des pièces}
\subsubsection{Listes et lien web}

\subsubsection{Datasheet des PDF}


\pagebreak
\section{Interface PC}

\subsection{Structure du programme}

\subsection{Explication des trames}

\subsection{Ordre de gestion des tâches}





\pagebreak
\section{Logiciel du SOC8200}
\subsection{Description du programme}

\subsection{Schéma bloc}
\subsubsection{Du code}

\subsubsection{Du script shell}

\subsection{Gestion des processus et du temps de CPU}

\subsection{Format et récupration des logs}

\subsection{Liste des tests et logiciels}






\pagebreak
\section{Logiciel de la station 1 et 2 et du bolide}

\subsection{La station no.1}

\subsection{La station no.2}

\subsection{Le bolide}

\subsection{Procédure de compilation sur IAR}

\subsection{Procédure de vérification}




\section{Logiciel du module PIC18F258}
\subsection{Description du fonctionnement du programme}

\subsection{Procédure de compilation sur MPLAB}

\subsection{Procédure de vérification}






\pagebreak
\section{Conclusion}

\subsection{Ce que le projet m'a apporté}
\subsubsection{Vincent Chouinard}

\subsubsection{Hicham Safoine}

\subsubsection{Gabriel Fortin-Bélanger}

\subsubsection{Louis-Norman Ang-Houle}



\subsection{Difficultés et corrections}
\subsubsection{Vincent Chouinard}

\subsubsection{Hicham Safoine}

\subsubsection{Gabriel Fortin-Bélanger}

\subsubsection{Louis-Norman Ang-Houle}


\subsection{Ce que j'ai aimé ou pas}
\subsubsection{Vincent Chouinard}

\subsubsection{Hicham Safoine}

\subsubsection{Gabriel Fortin-Bélanger}

\subsubsection{Louis-Norman Ang-Houle}
\end{spacing}
\end{document}

