\documentclass[10pt,a4paper,final]{article}

\usepackage{geometry}                            
\geometry{a4paper,total={210mm,297mm},left=10mm,right=10mm,top=20mm,bottom=20mm} 

\usepackage{fancyhdr}          % Met les pieds et têtes de pages
\pagestyle{fancy}

\usepackage[francais]{babel}   % Met les accents français
\usepackage[utf8]{inputenc}
\usepackage[pdftex]{graphicx}

\usepackage{graphicx}
\usepackage{framed}
\usepackage{tabularx}          % Liste les tableaux en table des matières
\usepackage{listings}
\usepackage{lipsum}
\usepackage{multicol}          % Permet le multicolone
\usepackage{xcolor}

\renewcommand\familydefault{\sfdefault}

\usepackage{tgheros}                
\usepackage[defaultmono]{droidmono} 
\usepackage{amsmath,amssymb,amsthm,textcomp}
%\usepackage{ubuntu}
   % Packages et dépendances nécessaires
\include{TeX/SourceCode} % Permet de mettre du code source dans le document

\begin{document}
\renewcommand\headrulewidth{0pt}
\fancyfoot[C]{ }






\begin{center}
\includegraphics[scale=1]{Figures/School_Logo.jpg}~\\[1cm]  
\textsc{\LARGE \'{E}lectronique Programmable et Robotique}\\[1.5cm]
\Large 247-6[1-2-3-4]7-LI\\[0.5cm]
{ \huge \bfseries Projet de 5$\mathbf{^{e}}$ session \\[0.4cm] }
\HRule \\[1.5cm]





\begin{multicols}{2}
\begin{flushleft}



\textbf{Étudiants:}\\

\bigskip

Vincent Chouinard\\
Hicham Safoine\\
Gabriel Fortin-Bélanger\\ 
Louis-Nomand Ang-Houle\\




\end{flushleft}
\vfill
\begin{flushright}

\textbf{Professeurs:}\\
\medskip
Ali Tadli\\
Alain Champagne\\
Stéphane Deschênes\\
Étienne Tremblay\\



\end{flushright}
\end{multicols}

\bigskip
\bigskip


\includegraphics[scale=0.6]{Figures/Picture_for_Title.jpg} 

\vfill
L'usine à gaz, et le gaz, c'est de l'air!
\bigskip

{\large \today}
\end{center}






\pagebreak
\begin{spacing}{1.24999}
\tableofcontents 
%\pagebreak
\listoffigures  
\listoftables 
\end{spacing}  
\pagebreak






\renewcommand\headrulewidth{1pt}
\fancyhead[L]{247-6[1-2-3-4]7-LI}
\fancyhead[R]{Projet de 5$\mathbf{^{e}}$ session}

\renewcommand\footrulewidth{1pt}
\fancyfoot[C]{\textbf{page \thepage}}
\fancyfoot[R]{\today}
\fancyfoot[L]{V.C., H.S., G.F-B., L-N.A-H.}
\begin{spacing}{1.5}

\section{Présentation du projet}

\subsection{Explication du projet}

\subsection{Schéma bloc du système}
\subsubsection{Bloc 1}

\subsubsection{Bloc 2}

\subsubsection{Bloc 3}

\subsubsection{Bloc 4}

\subsection{Liste des logiciels}
\begin{flushleft}
\HRule
\end{flushleft}
\begin{flushleft}

\begin{multicols}{3}

\textbf{Terminaux}
\begin{itemize}
\item[•]UART Master 1.0.3
\item[•]Serializ3r 1.0.2
\item[•]TerraTerm
\item[•]Putty
\item[•]GTKterm 0.99.7-rc1
\item[•]xTerminator
\item[•]CAPS
\item[•]tinyBootloader
\end{itemize}

\textbf{Gestion du projet}
\begin{itemize}
\item[•]MS Project 2012
\item[•]Git Hub
\end{itemize}

\textbf{Compilateurs et IDE}
\begin{itemize}
\item[•]Visual Studio 2013
\item[•]Visual Studio 2010
\item[•]IAR 8.20
\item[•]MPLAB
\end{itemize}

\textbf{Éditeur de texte}
\begin{itemize}
\item[•]Notepad++
\item[•]gedit
\item[•]medit 1.2.0
\end{itemize}
\textbf{Schémas électriques}

\begin{itemize}
\item[•]OrCAD 16.2
\end{itemize}

\bigskip

\textbf{Système d'exploitation}
\begin{itemize}
\item[•]Windows 7 SP1
\item[•]Windows 8.1
\item[•]Windows XP SP3
\item[•]Fedora 20
\item[•]CentOS
\item[•]Lubuntu 14.10
\end{itemize}

\textbf{Autres}
\begin{itemize}
\item[•]VMWare Workstation 10
\item[•]TeXmaker 4.3
\item[•]Dukto R6
\item[•]Dia
\end{itemize}

\end{multicols}
\end{flushleft}
\begin{flushleft}
\HRule
\end{flushleft}

\pagebreak
\subsection{Liste des trames}

\begin{table}[!ht]
\caption{Index des identifiant matériel CAN}
\medskip
\centering
\begin{tabular}{|l|c|}
\hline 
\textbf{Device} & \textbf{Identifiant matériel} \\ 
\hline 
Ordinateur & 000 \\ 
\hline 
SOC8200 & 001 \\ 
\hline 
Station 1 & 002 \\ 
\hline 
Station 2  & 003 \\ 
\hline 
Véhicule  & 004 \\ 
\hline 
\end{tabular} 
\label{tab:testtab1}
\end{table} 


\begin{table}[!ht]
\caption{Index des trames CAN}
\medskip
\centering
\begin{tabular}{|l|c|l|c|}
\hline 
\textbf{Fonctionnalité} & \textbf{Composante} & \textbf{Données} & \textbf{TimeStamp}\\
\hline 
Démarre le véhcule & 0x00 & 0x00 & TimeStamp\\ 
\hline 
Arrête le véhicule & 0x00 & 0x01 & TimeStamp\\ 
\hline 
Le véhicule est arrêté & 0x01 & 0x00 & TimeStamp\\ 
\hline 
Le véhicule est en marche& 0x01 & 0x01 & TimeStamp\\ 
\hline 
Le véhicule est hors circuit & 0x01 & 0x02 & TimeStamp\\ 
\hline 
Vitesse (0-100) & 0x02 & 0x00 à 0x64 & TimeStamp\\ 
\hline 
Battrie & 0x03 & 0x00 à 0x64 & TimeStamp\\ 
\hline 
Couleur du bloc & 0x04 & 0x00 à 0x02 & TimeStamp\\ 
\hline 
Poids du bloc & 0x05 & 0x00 à 0x64 & TimeStamp\\ 
\hline 
Envoyer l'heure & 0x06 &   & TimeStamp\\ 
\hline 
No. de la station & 0x07 & 0x00 à 0x02 & TimeStamp\\ 
\hline 
\end{tabular} 
\label{tab:testtab1}
\end{table}

\begin{table}[!ht]
\caption{Index des communication CAN}
\medskip
\centering
\begin{tabular}{|l|l|c|c|l|c|}
\hline 
\textbf{Émetteur} & \textbf{Action} & \textbf{Donnée envoyée} & \textbf{TimeStamp} & \textbf{Récepteur} & \textbf{Erreur retournée au PC}\\ 
\hline 
Ordinateur & Démarrer le véhicule & 004 00 00 & TimeStamp & Véhicule & 000 F1\\ 
\hline 
Ordinateur & Arrêter le véhicule & 004 00 01 & TimeStamp & Véhicule & 000 F2\\ 
\hline 
Véhicule & Dit: je suis arrêté & 000 01 00  & TimeStamp & Ordinateur & 000 F3\\
\hline 
Véhicule & Dit: j'avance & 000 01 01 & TimeStamp & Ordinateur  & 000 F4\\  
\hline 
Véhicule & Dit: je suis hors circuit & 000 01 02 & TimeStamp & Ordinateur & 000 F5\\ 
\hline 
Véhicule & Dit sa vitesse & 000 02 [00 à 64] & TimeStamp & Ordinateur & 000 F6\\ 
\hline 
Véhicule & Dit le niveau de sa battrie & 000 03 [00 à 64] & TimeStamp & Ordinateur & 000 F7\\ 
\hline 
Station 1 & Dit bloc = métal & 000 04 00 & TimeStamp & Ordinateur & 000 F8\\ 
\hline  
Station 1 & Dit bloc = orange & 000 04 01 & TimeStamp & Ordinateur & 000 F9\\
\hline 
Station 1 & Dit bloc = noir & 000 04 02 & TimeStamp & Ordinateur & 000 FA\\
\hline 
Station 1 & Dit le poid du bloc & 000 05 [00 à 64] & TimeStamp & Ordinateur & 000 FB\\ 
\hline 
Voiture & Dit qu'elle est à la station 1 & 000 07 00 & TimeStamp & Ordinateur & 000 FC\\ 
\hline 
Voiture & Dit qu'elle est à la station 2 & 000 07 01 & TimeStamp & Ordinateur & 000 FD\\
\hline 
Ordinateur & Envoie l'heure & 003 06 ? & TimeStamp & Station 1 & 000 FE\\ 
\hline 
\end{tabular} 
\label{tab:testtab1}
\end{table}
\begin{flushleft}
\textbf{Note:} S'il n'y a aucune erreur, le code retourné est 000 FF\\
\textbf{Note:} Il faut définir les TimeStamps\\
\textbf{Note:} La sttion no.1 relaie les données entre l'ordinateur et la station no.3 (pesage); entre l'ordinateur et la station no.2 (table festo) via xbee; entre la voiture et le PC via xbee.
\end{flushleft}
\pagebreak


\section{Le matériel}
\subsection{Bloc 1}

\subsection{Bloc 2}

\subsection{Bloc 3}

\subsection{Bloc 4}



\subsection{Explication des types de liens}



\subsection{Explication des trames}
\subsubsection{RS-232}

\subsubsection{CAN}

\subsubsection{XBEE}




\subsection{Liste des pièces}
\begin{center}
\HRule
\end{center}
\begin{multicols}{3}
\begin{itemize}
\item[•]Carte Dallas
\item[•]Carte uPSD
\item[•]SOC 8200
\item[•]PIC18Fmachin
\item[•]Carte d'extension SPI
\item[•]Carte d'extension I2C
\item[•]Carte CAN MCP2515
\item[•]XBEE
\item[•]Table FESTO
\item[•]Carte d'extension IO
\item[•]Carte connecteur DAC ADC
\item[•]
\item[•]
\item[•]
\item[•]
\item[•]
\item[•]
\item[•]
\item[•]
\end{itemize}
\end{multicols}
\begin{center}
\HRule
\end{center}


\subsubsection{Liens web}

\subsubsection{Datasheet des PDF}


\pagebreak
\section{Interface PC}

\subsection{Structure du programme}

\subsection{Explication des trames}

\subsection{Ordre de gestion des tâches}





\pagebreak
\section{Logiciel du SOC8200}
\subsection{Description du programme}

\subsection{Schéma bloc}
\subsubsection{Du code}

\subsubsection{Du script shell}

\subsection{Gestion des processus et du temps de CPU}

\subsection{Format et récupration des logs}

\subsection{Liste des tests et logiciels}






\pagebreak
\section{Logiciel de la station 1 et 2 et du bolide}

\subsection{La station no.1}

\subsection{La station no.2}

\subsection{Le bolide}

\subsection{Procédure de compilation sur IAR}

\subsection{Procédure de vérification}




\section{Logiciel du module PIC18F258}
\subsection{Description du fonctionnement du programme}

\subsection{Procédure de compilation sur MPLAB}

\subsection{Procédure de vérification}






\pagebreak
\section{Conclusion}

\subsection{Ce que le projet m'a apporté}
\subsubsection{Vincent Chouinard}

\subsubsection{Hicham Safoine}

\subsubsection{Gabriel Fortin-Bélanger}

\subsubsection{Louis-Norman Ang-Houle}



\subsection{Difficultés et corrections}
\subsubsection{Vincent Chouinard}

\subsubsection{Hicham Safoine}

\subsubsection{Gabriel Fortin-Bélanger}

\subsubsection{Louis-Norman Ang-Houle}


\subsection{Ce que j'ai aimé ou pas}
\subsubsection{Vincent Chouinard}

\subsubsection{Hicham Safoine}

\subsubsection{Gabriel Fortin-Bélanger}

\subsubsection{Louis-Norman Ang-Houle}
\end{spacing}
\end{document}

