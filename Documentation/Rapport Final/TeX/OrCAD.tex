
\begin{figure}[hbtp]
\section{Schémas OrCAD}
\caption{Schémas carte IO I2C, page 1}
\centering
\includegraphics[scale=0.87,angle=90]{Figures/OrCad/PAGE1.pdf}
\end{figure}

\begin{figure}[hbtp]
\caption{Schémas carte IO I2C, page 2}
\centering
\includegraphics[scale=0.87,angle=90]{Figures/OrCad/PAGE2.pdf}
\end{figure}

\begin{figure}[hbtp]
\caption{Schémas carte IO I2C, page 3}
\centering
\includegraphics[scale=0.87,angle=90]{Figures/OrCad/PAGE3.pdf}
\end{figure}

\begin{figure}[hbtp]
\caption{Schémas carte IO I2C, page 4}
\centering
\includegraphics[scale=0.87,angle=90]{Figures/OrCad/PAGE4.pdf}
\end{figure}

\vfill
\pagebreak

\begin{figure}[hbtp]
\section{Fichiers Gerbers}
Une carte d'extension, dont voici les images GERBER\footnote{Les couches présentées no sont pas à l'échelle}, à été réalisée avec OrCAD 16.2 et gravée à l'aide de la rutilante LPKF départementale.
\caption{Couche TOP}
\centering
\fbox{\includegraphics[scale=0.47]{Figures/OrCad/top.png}}

\bigskip
\bigskip
\bigskip
\bigskip
\bigskip

\caption{Couche BOT}
\fbox{\includegraphics[scale=0.47]{Figures/OrCad/bot.png}}
\end{figure}


\begin{figure}[hbtp]
\caption{Silk Screen TOP}
\centering
\fbox{\includegraphics[scale=0.47]{Figures/OrCad/sstop.png}}

\bigskip
\bigskip
\bigskip
\bigskip
\bigskip

\caption{Solder mask TOP}
\fbox{\includegraphics[scale=0.47]{Figures/OrCad/smt.png}}
\end{figure}



\begin{figure}[hbtp]
\caption{Drill}
\centering
\fbox{\includegraphics[scale=0.45]{Figures/OrCad/drill.png}}

\bigskip
\bigskip
\bigskip
\bigskip
\bigskip

%\caption{Correctif}
%\fbox{\includegraphics[scale=0.45]{Figures/OrCad/patch.jpg}}

\bigskip

\end{figure}