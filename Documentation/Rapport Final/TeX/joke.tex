
\textbf{Bonus}\\

\bigskip

\noindent $\mathbf{*54\cdot43.} \vdash:.\alpha,\beta\in1.\supset:\alpha\cap\beta=\Lambda.\equiv.\alpha\cup\beta\in2$\\ 
\indent\emph{Dem.}
\begin{flalign}\nonumber
\vdash .*54\cdot26.\supset\vdash:.\alpha=\iota'x.\beta=\iota'y.\supset:\alpha\cup\beta\in2.&\equiv.x\neq y.\\\nonumber
[*51\cdot 231]\hspace{4.7cm}\hspace{1cm} & \equiv.t'x\cap\iota'y=\Lambda.\\
[*13\cdot 12]\hspace{4.88cm}\hspace{1cm} & \equiv.\alpha\cap\beta=\Lambda \\\nonumber
\vdash.(1).*11\cdot11\cdot35.\supset\hspace{2.88cm}\hspace{1cm}\\
\vdash:.(\exists x,y).\alpha=\iota'x.\beta=\iota'y.\supset:\alpha\cup\beta\in2.&\equiv.\alpha\cap\beta=\Lambda\\\nonumber
\vdash.(2).*11\cdot54.*52\cdot1.\supset\vdash.Prop\hspace{1.09cm}\hspace{1cm}\end{flalign}
\indent From this proposition it will follow, when arithmetical addition has been defined, that $1 + 1 = 2$.


\bigskip


\bigskip


\bigskip


\bigskip

\begin{flushleft}
Cette démonstration mathématique prouve hors de tout doute que $1 + 1 = 2$.\\ Si les mathématiques sont vrais, alors notre projet devrait aller.

\end{flushleft}